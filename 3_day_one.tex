% Source http://vgperson.tumblr.com/post/130927788020/lieat-the-novel-prologues-part-i
%   \section{The Lie-Eating Dragon and the Forgotten-Color Songstress}
%	\subsection*{Ngày 1}

"Bố ơi, cái gì thế ?", Efi giật mạnh tay áo tôi và hỏi. Tôi làm ngơ, nhìn chầm chầm ra cửa sổ. 

Ngay cả khi bọn tôi đi tàu hỏa, tôi phải nói là nó chạy rất chậm. Chiếc xe ngựa vừa qua mặt chúng tôi khi nãy còn nhanh hơn. Mớ sắt vụn này chỉ giỏi giằng xóc.  

"Bố ơi, cái gì vậy !!!"

"Hừ, câm mồm, nhóc. Và đừng gọi tao là bố."

Tôi bực mình quá nên nói nặng với Efi. Nghe không giống như tôi mọi ngày. Nhưng hãy thông cảm cho một thằng như tôi mắc kẹt với đứa trẻ.

"Con không phải "nhóc". Con là Efi !!!"

...Tôi buộc phải nói chuyện thế này mỗi ngày, mỗi giờ. 

Cái ý tưởng mà trẻ con dễ thương đúng là được bọn mộng mơ nghĩ ra. Bọn không bao giờ phải giữ trẻ hết. Không thể thấy cái "dễ thương" trong lũ trẻ, "dễ ghét" thì đúng hơn.

Tôi thở dài, như thể xua tan cái cảm giác không lối thoát này.

"Nó là gì vậy ? Cái dài dài ốm ốm quấn quanh cổ bố"

Efi mắt long lanh, chỉ tay vào cái cổ của tôi. Cảnh hỏi đáp này sẽ kéo dài đến bao giờ ?

"... Đây là khăn quàng cổ. Mày quàng nó vào để khỏi lạnh, hoặc để cho đẹp."

"Đẹp ! Efi muốn đẹp"

Tôi vừa nói xong thì Efi nắm lấy khăn quàng cổ, kéo mạnh. Quá nhanh, tôi phản xạ không kịp, và bị nó siết cổ. Tôi kêu lên.

Tôi thấy mệt muốn chết vì cuộc trò chuyện hồi nãy, giờ lại còn hành xác nữa. Tôi chẳng mong đợi gì từ thế giới này, nhỉ ...

"Aaaaaaah ! Dừng lại, nhóc con ! Ngồi yên dùm cái !!"

Tôi giật tay nó ra (khi nó đang cố siết cổ tôi lần nữa), và tháo cái khăn ra, và quàng vào Efi. 

...Cái khăn không phục vụ mục đích giữ ấm hay thời trang gì cả. Chỉ vì Efi thích cái khăn ... 

"E he he he! Giờ Efi đẹp nè !"

... Tôi nghĩ sẽ không có chuyện gì xảy ra, miễn sao nó hài lòng.

Công việc chưa bắt đầu, mà sao tôi thấy chóng cả mặt ? Tôi không thể không lo lắng cho những gì còn phía trước.  \\
% Work hadn’t even started, so why was I feeling dizzy here of all places? I couldn’t help but worry for what lied ahead.
% paragraph break

\begin{center}
	*****
\end{center}
%It had been about a month since my shocking encounter with Efi.

Thấm thoát đã 1 tháng kể từ cuộc gặp gỡ định mệnh với Efi. 

Từ hôm đó, tôi đã dò hỏi trực tiếp Efi và tìm hiểu càng nhiều càng tốt, nhưng tôi chỉ biết Efi là con rồng ăn nói dối. Nó còn không biết tại sao nó được sinh ra. 

Lúc đầu, tôi vui mừng vì có trong tay một bé rồng, vì mọi người đều muốn. Nhưng tôi bắt đầu thấy hối hận khi trải qua những tháng ngày mệt mỏi.

Nếu tôi bán nó để lấy một đống tiền khi nó còn là quả trứng, mọi chuyện chắc tốt hơn. 

"Bố ơi, sao bố mệt mỏi vậy ?"

Tại mày đó. Tôi kiềm chế không nói ra. Ồ, có thể là do tôi chẳng còn sức để nói. Thay vào đó, tôi chỉ khẻ rên một tiếng.

Chắc tôi nên ngủ một tí. Bỗng dưng tôi cảm thấy cái điện thoại để trong túi nó rung. Tôi lẩm bẩm "Ai nhỉ?"

"Ồ, sao căng vậy ? Ở đó thế nào ?"

"... Kẻ buôn tin đội mũ phải hông ?"

Người phụ nữ sống bằng nghề buôn tin, mua và bán thông tin, vừa gọi tôi. Bả luôn đội cái mũ trông khá là nổi, nên tôi gọi là "Kẻ buôn tin đội mũ".

Tất nhiên là lúc này, tôi chẳng thể thấy cái mũ xa hoa của ả qua điện thoại. 

"Hừm, nghe có vẻ chú đang trên xe lửa ? Nó chạy chậm như rùa hả ? Chú đang đổi đời ?"

"Hy vọng đéo."

Kẻ buôn tin này rất giỏi, nhưng cũng khá trẻ con. Tôi chẳng giỏi nói chuyện với ả, nhưng ả có vẻ thích Efi. 

Ồ, Efi, tôi nhìn nó, và thấy nó đang ngủ ngon lành với cái khăn quàng cổ quấn quanh người. 

"Chán thật. Dạo này chọc chú chẳng có tác dụng gì, chú biết ấy ? Nhân tiện, lần này chú lấy biệt danh gì ? Chaude ? Duke ? "

"Mấy tên ấy cũ rồi... Lần này, Al."

Thật may là Efi ngủ rồi, nên giờ tôi được yên tĩnh một mình. Tôi cần lập kế hoạch rõ ràng trước khi tới nơi. Một sai lầm nhỏ sẽ tốn rất nhiều thời gian, công sức và tiền bạc để đến đây. Tôi cần phải tránh điều đó bằng mọi giá.

"Được rồi, vậy thì Al. Tôi sẽ cung cấp tin cho chú khi tôi vẫn còn cơ hội."

Rồi bả bắt đầu giải thích. 

Tôi đi đến thị trấn Indigo theo yêu cầu của bà buôn tin đội mũ để xác thực về một số tin đồn. 

Tin đồn ấy chính là truyền thuyết về con chim lam. Con chim biến điều ước của con người thành sự thật.
 
"Chú nên bắt luôn con chim, nếu chú có thể ... Sao ?"

"Tôi chẳng thể nào nói rằng tôi bắt được hay không đến khi tôi biết con chim ấy tồn tại."

Bà buôn tin lải nhải qua điện thoại.

"Chú đang nói về tin của tôi, ấy nhỉ ? Được rồi, tôi không chắc làm thế nào cho con chim nó xuất hiện. Nhưng mà, hỏi cư dân ở đó là cách nhanh nhất để biết."

"Rồi, hiểu"

Cuộc trò chuyện giữa chúng tôi bỗng dưng kết thúc, và tiếng thông báo xe lửa đã đến thị trấn Indigo oang oang. Tôi chào tạm biệt ả rồi cúp máy. Sau đó, tôi đánh thức Efi dậy khi nó đang ngủ say như chết. 

"Con no rồi ....", nó lầm bầm, chải nước dãi với vẻ mặt thỏa mãn. Tôi thưởng nó một cái búng vào trán. \\













