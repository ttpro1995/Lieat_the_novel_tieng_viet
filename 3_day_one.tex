% Source http://vgperson.tumblr.com/post/130927788020/lieat-the-novel-prologues-part-i
%   \section{The Lie-Eating Dragon and the Forgotten-Color Songstress}
%	\subsection*{Ngày 1}

"Bố ơi, cái gì thế ?", Efi giật mạnh tay áo tôi và hỏi. Tôi làm ngơ, nhìn chầm chầm ra cửa sổ. 

Ngay cả khi bọn tôi đi tàu hỏa, tôi phải nói là nó chạy rất chậm. Chiếc xe ngựa vừa qua mặt chúng tôi khi nãy còn nhanh hơn. Mớ sắt vụn này chỉ giỏi giằng xóc.  

"Bố ơi, cái gì vậy !!!"

"Hừ, câm mồm, nhóc. Và đừng gọi tao là bố."

Tôi bực mình quá nên nói nặng với Efi. Nghe không giống như tôi mọi ngày. Nhưng hãy thông cảm cho một thằng như tôi mắc kẹt với đứa trẻ.

"Con không phải "nhóc". Con là Efi !!!"

...Tôi buộc phải nói chuyện thế này mỗi ngày, mỗi giờ. 

Cái ý tưởng mà trẻ con dễ thương đúng là được bọn mộng mơ nghĩ ra. Bọn không bao giờ phải giữ trẻ hết. Không thể thấy cái "dễ thương" trong lũ trẻ, "dễ ghét" thì đúng hơn.

Tôi thở dài, như thể xua tan cái cảm giác không lối thoát này.

"Nó là gì vậy ? Cái dài dài ốm ốm quấn quanh cổ bố"

Efi mắt long lanh, chỉ tay vào cái cổ của tôi. Cảnh hỏi đáp này sẽ kéo dài đến bao giờ ?

"... Đây là khăn quàng cổ. Mày quàng nó vào để khỏi lạnh, hoặc để cho đẹp."

"Đẹp ! Efi muốn đẹp"

Tôi vừa nói xong thì Efi nắm lấy khăn quàng cổ, kéo mạnh. Quá nhanh, tôi phản xạ không kịp, và bị nó siết cổ. Tôi kêu lên.

Tôi thấy mệt muốn chết vì cuộc trò chuyện hồi nãy, giờ lại còn hành xác nữa. Tôi chẳng mong đợi gì từ thế giới này, nhỉ ...

"Aaaaaaah ! Dừng lại, nhóc con ! Ngồi yên dùm cái !!"

Tôi giật tay nó ra (khi nó đang cố siết cổ tôi lần nữa), và tháo cái khăn ra, và quàng vào Efi. 

...Cái khăn không phục vụ mục đích giữ ấm hay thời trang gì cả. Chỉ vì Efi thích cái khăn ... 

"E he he he! Giờ Efi đẹp nè !"

... Tôi nghĩ sẽ không có chuyện gì xảy ra, miễn sao nó hài lòng.

Công việc chưa bắt đầu, mà sao tôi thấy chóng cả mặt ? Tôi không thể không lo lắng cho những gì còn phía trước.  \\
% Work hadn’t even started, so why was I feeling dizzy here of all places? I couldn’t help but worry for what lied ahead.
% paragraph break
\begin{center}
	******
\end{center}
%It had been about a month since my shocking encounter with Efi.

Thấm thoát đã 1 tháng kể từ cuộc gặp gỡ định mệnh với Efi. 

Từ hôm đó, tôi đã dò hỏi trực tiếp Efi và tìm hiểu càng nhiều càng tốt, nhưng tôi chỉ biết Efi là con rồng ăn nói dối. Nó còn không biết tại sao nó được sinh ra. 

Lúc đầu, tôi vui mừng vì có trong tay một bé rồng, vì mọi người đều muốn. Nhưng tôi bắt đầu thấy hối hận khi trải qua những tháng ngày mệt mỏi.

Nếu tôi bán nó để lấy một đống tiền khi nó còn là quả trứng, mọi chuyện chắc tốt hơn. 

"Bố ơi, sao bố mệt mỏi vậy ?"

Tại mày đó. Tôi kiềm chế không nói ra. Ồ, có thể là do tôi chẳng còn sức để nói. Thay vào đó, tôi chỉ khẻ rên một tiếng.

Chắc tôi nên ngủ một tí. Bỗng dưng tôi cảm thấy cái điện thoại để trong túi nó rung. Tôi lẩm bẩm "Ai nhỉ?"

"Ồ, sao căng vậy ? Ở đó thế nào ?"

"... Kẻ buôn tin đội mũ phải hông ?"

Người phụ nữ sống bằng nghề buôn tin, mua và bán thông tin, vừa gọi tôi. Bả luôn đội cái mũ trông khá là nổi, nên tôi gọi là "Kẻ buôn tin đội mũ".

Tất nhiên là lúc này, tôi chẳng thể thấy cái mũ xa hoa của ả qua điện thoại. 

"Hừm, nghe có vẻ chú đang trên xe lửa ? Nó chạy chậm như rùa hả ? Chú đang đổi đời ?"

"Hy vọng đéo."

Kẻ buôn tin này rất giỏi, nhưng cũng khá trẻ con. Tôi chẳng giỏi nói chuyện với ả, nhưng ả có vẻ thích Efi. 

Ồ, Efi, tôi nhìn nó, và thấy nó đang ngủ ngon lành với cái khăn quàng cổ quấn quanh người. 

"Chán thật. Dạo này chọc chú chẳng có tác dụng gì, chú biết ấy ? Nhân tiện, lần này chú lấy biệt danh gì ? Chaude ? Duke ? "

"Mấy tên ấy cũ rồi... Lần này, Al."

Thật may là Efi ngủ rồi, nên giờ tôi được yên tĩnh một mình. Tôi cần lập kế hoạch rõ ràng trước khi tới nơi. Một sai lầm nhỏ sẽ tốn rất nhiều thời gian, công sức và tiền bạc để đến đây. Tôi cần phải tránh điều đó bằng mọi giá.

"Được rồi, vậy thì Al. Tôi sẽ cung cấp tin cho chú khi tôi vẫn còn cơ hội."

Rồi bả bắt đầu giải thích. 

Tôi đi đến thị trấn Indigo theo yêu cầu của bà buôn tin đội mũ để xác thực về một số tin đồn. 

Tin đồn ấy chính là truyền thuyết về con chim lam. Con chim biến điều ước của con người thành sự thật.
 
"Chú nên bắt luôn con chim, nếu chú có thể ... Sao ?"

"Tôi chẳng thể nào nói rằng tôi bắt được hay không đến khi tôi biết con chim ấy tồn tại."

Bà buôn tin lải nhải qua điện thoại.

"Chú đang nói về tin của tôi, ấy nhỉ ? Được rồi, tôi không chắc làm thế nào cho con chim nó xuất hiện. Nhưng mà, hỏi cư dân ở đó là cách nhanh nhất để biết."

"Rồi, hiểu"

Cuộc trò chuyện giữa chúng tôi bỗng dưng kết thúc, và tiếng thông báo xe lửa đã đến thị trấn Indigo oang oang. Tôi chào tạm biệt ả rồi cúp máy. Sau đó, tôi đánh thức Efi dậy khi nó đang ngủ say như chết. 

"Con no rồi ....", nó lầm bầm, chải nước dãi với vẻ mặt thỏa mãn. Tôi thưởng nó một cái búng vào trán. \\


\begin{center}
	******
\end{center}


% “It hurts...” Clutching to my sleeve, Efi rubbed her forehead with the other hand.

"Đau quá ..." Nó nắm chặt tay áo tôi, tay còn lại nó xoa xoa cái trán. 
Tôi không nghĩ là tôi búng mạnh đến vậy. Tôi bắt đầu lo khi nó cứ làm thế này. Chắc tôi búng mạnh thật. 

"Đau quá ..." Nó liếc mắt nhìn tôi.

À, ra là vậy. Nó đang muốn tôi để ý nó. Lúc đầu tôi chẳng biết ý nó, nhưng sau một tháng chung sống thì, vâng, tôi có thể thấy nó muốn đang.

"Này, tao đã bảo cả triệu lần khi còn trên tàu lửa là, đừng chạy trước tao."

"Này, đợi đã, đi chậm lại !!!"

Tôi chẳng còn thời gian để ý nó. Đầu tiên, tôi cần tìm chỗ trọ đã. Tàu hỏa đến trể nên tôi khá vội.

"Vâng, tàu hỏa  ? Bố có phải trả tiền để đi tàu hỏa không ?" Nó nhận ra rằng cách cũ không hiệu quả để gây sự chú ý của tôi, nên nó nói chuyện với tôi như mọi lần.

"Rõ ràng mà. Nếu mày nghĩ cái gì đó dễ, có nghĩa là nó tốn tiền"

"Nhưng hồi nãy bố nói bố chẳng còn tiền mà."

Sao nó phải nhớ cái này nhỉ ?

"... Nghe này, Efi. Không phải đéo có tiền là vấn đề có thể giải quyết được ngay."

Nó gật đầu hào hứng.

"Nhưng với cái khác, mày có thể làm gì đó."

"Sao ... ?"

Tôi móc trong túi ra, cái vé giả mà tôi làm tối qua.

"Nếu mày không có, thì mày tự làm ra nó."\\

Chúng tôi đi bộ một lúc, nhưng thị trấn Indigo khá to và đông cư dân. Cộng thêm, ngoài trừ mấy người mà tôi thường thấy trong bóng tối, cư dân ở đây khá tấp nập... Một vài người khiến tôi chú ý, nhưng tôi cần tìm chỗ trọ đã.

Ở trên quảng trường với cái dài phun nước, tôi quyết định sẽ bắt chuyện với một cư dân đang ngồi nghỉ trên băng ghế dài. Hồi nãy, khi vừa xuống tàu, tôi lấy lại cái khăn quàng cổ từ Efi, chỉnh sửa lại giọng nói và khuôn mặt cho phù hợp với thành phố này. 

"Xin chào, tôi có thể hỏi ông một vài thứ không ?"

Ông từ từ ngẩn đầu, nhìn tôi. Đó là một người đàn ông trung niên với bộ râu rậm rạp một cách nổi bật. Khi thấy vừa thấy tôi, nét mặt ông rạng rỡ, ông đứng dậy.

"Ồ, cậu là một khách du lịch hả ? Sao ? Cậu thấy thị trấn thế nào ?"

"À, không khí trong lành. Hơn nữa, cư dân ở đây trông thật vui vẻ."
   
"Không nghi ngờ gì ! Sống ở đây khiến cậu hạnh phúc. Mọi người gọi đây là thị trấn hạnh phúc ! Tôi hi vọng cậu có thể mang về thật nhiều hạnh phúc, chàng trai trẻ."

Ông ấy quàng tay qua vai tôi, nở nụ cười nồng nhiệt. Ông ấy đè mạnh hơn tôi nghĩ, và cơ thể tôi kêu răn rắc. Tôi liếc nhìn Efi, khi nó chuẩn bị nói gì đấy, và ra hiệu cho nó im lặng.

Rồi tôi quay nhìn ông già lần nữa, nở nụ cười.

"Nhân tiện, tôi nên ở trọ chỗ nào trong thị trấn này. Ông thấy đấy, giờ tôi chưa tìm ra ..."

Người đàn ông trung niên gỡ tay ra khỏi vai tôi, và gãi đầu.

"Hừm, để xem. Theo tôi biết, hầu hết nhà nghỉ kín khách cả, hoặc đóng cửa, nên chỉ ... à !"

Ông ta vỗ tay trước ngực, như vừa nghĩ ra gì đó.

"Phải rồi, có một căn biệt thự sau thị trấn, nơi một cô gái trẻ sinh sống. Cô ấy là ca sĩ. Cậu gặp cô ấy xem sao ?"

"Đó là ... nhà riêng ?"

"Vâng, nhưng cô bé rất tốt bụng. Đừng lo, cô ấy sẽ cho anh ở nhờ."

Không có vấn đề cụ thế nào khi ở nhà riêng, nhưng công việc của một "kẻ lừa đảo" thì ở nhà nghỉ tiện hơn nhiều, vì có nhiều người ở đó, và tất cả. Nhưng nếu chẳng còn nhà nghỉ, thì tôi đoán là chẳng còn lựa chọn nào khác.

"... Hiểu rồi. Tôi sẽ đi hỏi thử. Ở phía sau thị trấn, phải không ?"

"Đúng vậy. Đi thẳng đường chính từ quảng trường và cậu sẽ tìm thấy căn biệt thự. Sự thật là, lúc này là lúc nữ ca sĩ ấy đang hát. Nếu cậu bị lạc, hãy theo tiếng hát của cô ấy".

Tôi cám ơn ông già và đi.

Khi đi được vài phút, Efi lại kéo tay áo tôi và hỏi, "Bố ơi, sao mỗi khi bố nói chuyện với cư dân ở thị trấn khác nhau, bố là bố khác ?"

"Không phải bố, tao là Al. Đừng gọi tao là bố, hiểu hông ?"

"Đó, giọng bố cũng đổi luôn. Và khuôn mặt nữa. Tại sao vậy ?"

Nó đang kéo tay áo tôi. Vẻ ngoài là thứ đầu tiên mà bạn phải để ý khi bạn đang lừa người ta. Lời nói của kẻ dở hơi với cái áo không phù hợp không bao giờ đáng tin.

Tôi búng vào trán Efi một cái, nó kêu "ui" rồi ngoan ngoãn đi theo.

Tôi dắt Efi ra khỏi quãng trường. Khi tôi thấy trên đường không có nhiều người, tôi bắt đầu trả lời câu hỏi khi nãy của nó. 

"Nghe này. Con người ... Khi ta gặp người nào đó lần đầu, ta đánh giá họ qua vẻ bề ngoài. Tiếp theo là giọng điệu. Cái gì ở trong, thì, để cuối cùng."

"Nhưng giờ bố đang quàng khăn che kín mặt ..."

"Tao không rảnh để lừa người ta trừ khi nó cần thiết". Tôi kéo khăng quàng lên thêm một tí nữa.

Lâu rồi, bà buôn tin đội mũ nói khuôn mặt thật của tôi, mặt mà tôi được sinh ra, không thích hợp cho việc xã giao. Từ đó, tôi giấu cảm xúc của tôi khi nói chuyện với người khác. Chắc chắn, tôi biết rằng tôi có ánh mắt khá nham hiểm, nhưng nó có đáng sợ đến vậy ?

"Hừm, Nhưng nó có nghĩa là, nói dối, phải không Al ?" Efi hỏi với vẻ mặt bất mãn.

"Nói dối chưa được gọi là nói dối khi người ta chưa biết.". Tôi trả lời, và nó nghiên đầu như thể hiểu ít hơn khi nãy.

"Nhưng có phải nói dối là xấu ? Anh lừa gạt mọi người, nên ... anh là người xấu ?"

Trong sáng đến mức này. Và không hề có ý xấu nào.

Trẻ con rất khó để xử với những thứ này. Đặc biệt là đối với kẻ kiếm sống bằng việc lừa gạt người ta, thật khó xử.

"Al... vậy anh là người xấu hả ?", nó lo lắng hỏi lại lần nữa.

Tôi gãi đầu mấy cái, và nói Efi cái ý mà tôi vừa nghĩ ra.

"Chắc chắn, tao là người xấu. Nhưng bọn bị lừa cũng vậy."

\begin{center}
	******
\end{center}

% After walking straight down the path for a while, a pretty sound came into earshot, like a bird chirping from far away. No... It was a human singing














