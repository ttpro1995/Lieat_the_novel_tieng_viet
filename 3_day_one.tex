% Source http://vgperson.tumblr.com/post/130927788020/lieat-the-novel-prologues-part-i
%   \section{The Lie-Eating Dragon and the Forgotten-Color Songstress}
%	\subsection*{Ngày 1}

"Bố ơi, cái gì thế ?", Efi giật mạnh tay áo tôi và hỏi. Tôi làm ngơ, nhìn chầm chầm ra cửa sổ. 

Ngay cả khi bọn tôi đi tàu hỏa, tôi phải nói là nó chạy rất chậm. Chiếc xe ngựa vừa qua mặt chúng tôi khi nãy còn nhanh hơn. Mớ sắt vụn này chỉ giỏi giằng xóc.  

"Bố ơi, cái gì vậy !!!"

"Hừ, câm mồm, nhóc. Và đừng gọi tao là bố."

Tôi bực mình quá nên nói nặng với Efi. Nghe không giống như tôi mọi ngày. Nhưng hãy thông cảm cho một thằng như tôi mắc kẹt với đứa trẻ.

"Con không phải "nhóc". Con là Efi !!!"

...Tôi buộc phải nói chuyện thế này mỗi ngày, mỗi giờ. 

Cái ý tưởng mà trẻ con dễ thương đúng là được bọn mộng mơ nghĩ ra. Bọn không bao giờ phải giữ trẻ hết. Không thể thấy cái "dễ thương" trong lũ trẻ, "dễ ghét" thì đúng hơn.

Tôi thở dài, như thể xua tan cái cảm giác không lối thoát này.

"Nó là gì vậy ? Cái dài dài ốm ốm quấn quanh cổ bố"

Efi mắt long lanh, chỉ tay vào cái cổ của tôi. Cảnh hỏi đáp này sẽ kéo dài đến bao giờ ?

"... Đây là khăn quàng cổ. Mày quàng nó vào để khỏi lạnh, hoặc để cho đẹp."

"Đẹp ! Efi muốn đẹp"

Tôi vừa nói xong thì Efi nắm lấy khăn quàng cổ, kéo mạnh. Quá nhanh, tôi phản xạ không kịp, và bị nó siết cổ. Tôi kêu lên.

Tôi thấy mệt muốn chết vì cuộc trò chuyện hồi nãy, giờ lại còn hành xác nữa. Tôi chẳng mong đợi gì từ thế giới này, nhỉ ...

"Aaaaaaah ! Dừng lại, nhóc con ! Ngồi yên dùm cái !!"

Tôi giật tay nó ra (khi nó đang cố siết cổ tôi lần nữa), và tháo cái khăn ra, và quàng vào Efi. 

...Cái khăn không phục vụ mục đích giữ ấm hay thời trang gì cả. Chỉ vì Efi thích cái khăn ... 

"E he he he! Giờ Efi đẹp nè !"

... Tôi nghĩ sẽ không có chuyện gì xảy ra, miễn sao nó hài lòng.

Công việc chưa bắt đầu, mà sao tôi thấy chóng cả mặt ? Tôi không thể không lo lắng cho những gì còn phía trước.  
% Work hadn’t even started, so why was I feeling dizzy here of all places? I couldn’t help but worry for what lied ahead.
% paragraph break
\begin{center}
	******
\end{center}
%It had been about a month since my shocking encounter with Efi.

Thấm thoát đã 1 tháng kể từ cuộc gặp gỡ định mệnh với Efi. 

Từ hôm đó, tôi đã dò hỏi trực tiếp Efi và tìm hiểu càng nhiều càng tốt, nhưng tôi chỉ biết Efi là con rồng ăn nói dối. Nó còn không biết tại sao nó được sinh ra. 

Lúc đầu, tôi vui mừng vì có trong tay một bé rồng, vì mọi người đều muốn. Nhưng tôi bắt đầu thấy hối hận khi trải qua những tháng ngày mệt mỏi.

Nếu tôi bán nó để lấy một đống tiền khi nó còn là quả trứng, mọi chuyện chắc tốt hơn. 

"Bố ơi, sao bố mệt mỏi vậy ?"

Tại mày đó. Tôi kiềm chế không nói ra. Ồ, có thể là do tôi chẳng còn sức để nói. Thay vào đó, tôi chỉ khẻ rên một tiếng.

Chắc tôi nên ngủ một tí. Bỗng dưng tôi cảm thấy cái điện thoại để trong túi nó rung. Tôi lẩm bẩm "Ai nhỉ?"

"Ồ, sao căng vậy ? Ở đó thế nào ?"

"... Kẻ buôn tin đội mũ phải hông ?"

Người phụ nữ sống bằng nghề buôn tin, mua và bán thông tin, vừa gọi tôi. Bả luôn đội cái mũ trông khá là nổi, nên tôi gọi là "Kẻ buôn tin đội mũ".

Tất nhiên là lúc này, tôi chẳng thể thấy cái mũ xa hoa của ả qua điện thoại. 

"Hừm, nghe có vẻ chú đang trên xe lửa ? Nó chạy chậm như rùa hả ? Chú đang đổi đời ?"

"Hy vọng đéo."

Kẻ buôn tin này rất giỏi, nhưng cũng khá trẻ con. Tôi chẳng giỏi nói chuyện với ả, nhưng ả có vẻ thích Efi. 

Ồ, Efi, tôi nhìn nó, và thấy nó đang ngủ ngon lành với cái khăn quàng cổ quấn quanh người. 

"Chán thật. Dạo này chọc chú chẳng có tác dụng gì, chú biết ấy ? Nhân tiện, lần này chú lấy biệt danh gì ? Chaude ? Duke ? "

"Mấy tên ấy cũ rồi... Lần này, Al."

Thật may là Efi ngủ rồi, nên giờ tôi được yên tĩnh một mình. Tôi cần lập kế hoạch rõ ràng trước khi tới nơi. Một sai lầm nhỏ sẽ tốn rất nhiều thời gian, công sức và tiền bạc để đến đây. Tôi cần phải tránh điều đó bằng mọi giá.

"Được rồi, vậy thì Al. Tôi sẽ cung cấp tin cho chú khi tôi vẫn còn cơ hội."

Rồi bả bắt đầu giải thích. 

Tôi đi đến thị trấn Indigo theo yêu cầu của bà buôn tin đội mũ để xác thực về một số tin đồn. 

Tin đồn ấy chính là truyền thuyết về con chim lam. Con chim biến điều ước của con người thành sự thật.
 
"Chú nên bắt luôn con chim, nếu chú có thể ... Sao ?"

"Tôi chẳng thể nào nói rằng tôi bắt được hay không đến khi tôi biết con chim ấy tồn tại."

Bà buôn tin lải nhải qua điện thoại.

"Chú đang nói về tin của tôi, ấy nhỉ ? Được rồi, tôi không chắc làm thế nào cho con chim nó xuất hiện. Nhưng mà, hỏi cư dân ở đó là cách nhanh nhất để biết."

"Rồi, hiểu"

Cuộc trò chuyện giữa chúng tôi bỗng dưng kết thúc, và tiếng thông báo xe lửa đã đến thị trấn Indigo oang oang. Tôi chào tạm biệt ả rồi cúp máy. Sau đó, tôi đánh thức Efi dậy khi nó đang ngủ say như chết. 

"Con no rồi ....", nó lầm bầm, chải nước dãi với vẻ mặt thỏa mãn. Tôi thưởng nó một cái búng vào trán. \\


\begin{center}
	******
\end{center}


% “It hurts...” Clutching to my sleeve, Efi rubbed her forehead with the other hand.

"Đau quá ..." Nó nắm chặt tay áo tôi, tay còn lại nó xoa xoa cái trán. 
Tôi không nghĩ là tôi búng mạnh đến vậy. Tôi bắt đầu lo khi nó cứ làm thế này. Chắc tôi búng mạnh thật. 

"Đau quá ..." Nó liếc mắt nhìn tôi.

À, ra là vậy. Nó đang muốn tôi để ý nó. Lúc đầu tôi chẳng biết ý nó, nhưng sau một tháng chung sống thì, vâng, tôi có thể thấy nó muốn đang.

"Này, tao đã bảo cả triệu lần khi còn trên tàu lửa là, đừng chạy trước tao."

"Này, đợi đã, đi chậm lại !!!"

Tôi chẳng còn thời gian để ý nó. Đầu tiên, tôi cần tìm chỗ trọ đã. Tàu hỏa đến trể nên tôi khá vội.

"Vâng, tàu hỏa  ? Bố có phải trả tiền để đi tàu hỏa không ?" Nó nhận ra rằng cách cũ không hiệu quả để gây sự chú ý của tôi, nên nó nói chuyện với tôi như mọi lần.

"Rõ ràng mà. Nếu mày nghĩ cái gì đó dễ, có nghĩa là nó tốn tiền"

"Nhưng hồi nãy bố nói bố chẳng còn tiền mà."

Sao nó phải nhớ cái này nhỉ ?

"... Nghe này, Efi. Không phải đéo có tiền là vấn đề có thể giải quyết được ngay."

Nó gật đầu hào hứng.

"Nhưng với cái khác, mày có thể làm gì đó."

"Sao ... ?"

Tôi móc trong túi ra, cái vé giả mà tôi làm tối qua.

"Nếu mày không có, thì mày tự làm ra nó."\\

Chúng tôi đi bộ một lúc, nhưng thị trấn Indigo khá to và đông cư dân. Cộng thêm, ngoài trừ mấy người mà tôi thường thấy trong bóng tối, cư dân ở đây khá tấp nập... Một vài người khiến tôi chú ý, nhưng tôi cần tìm chỗ trọ đã.

Ở trên quảng trường với cái dài phun nước, tôi quyết định sẽ bắt chuyện với một cư dân đang ngồi nghỉ trên băng ghế dài. Hồi nãy, khi vừa xuống tàu, tôi lấy lại cái khăn quàng cổ từ Efi, chỉnh sửa lại giọng nói và khuôn mặt cho phù hợp với thành phố này. 

"Xin chào, tôi có thể hỏi ông một vài thứ không ?"

Ông từ từ ngẩn đầu, nhìn tôi. Đó là một người đàn ông trung niên với bộ râu rậm rạp một cách nổi bật. Khi thấy vừa thấy tôi, nét mặt ông rạng rỡ, ông đứng dậy.

"Ồ, cậu là một khách du lịch hả ? Sao ? Cậu thấy thị trấn thế nào ?"

"À, không khí trong lành. Hơn nữa, cư dân ở đây trông thật vui vẻ."
   
"Không nghi ngờ gì ! Sống ở đây khiến cậu hạnh phúc. Mọi người gọi đây là thị trấn hạnh phúc ! Tôi hi vọng cậu có thể mang về thật nhiều hạnh phúc, chàng trai trẻ."

Ông ấy quàng tay qua vai tôi, nở nụ cười nồng nhiệt. Ông ấy đè mạnh hơn tôi nghĩ, và cơ thể tôi kêu răn rắc. Tôi liếc nhìn Efi, khi nó chuẩn bị nói gì đấy, và ra hiệu cho nó im lặng.

Rồi tôi quay nhìn ông già lần nữa, nở nụ cười.

"Nhân tiện, tôi nên ở trọ chỗ nào trong thị trấn này. Ông thấy đấy, giờ tôi chưa tìm ra ..."

Người đàn ông trung niên gỡ tay ra khỏi vai tôi, và gãi đầu.

"Hừm, để xem. Theo tôi biết, hầu hết nhà nghỉ kín khách cả, hoặc đóng cửa, nên chỉ ... à !"

Ông ta vỗ tay trước ngực, như vừa nghĩ ra gì đó.

"Phải rồi, có một căn biệt thự sau thị trấn, nơi một cô gái trẻ sinh sống. Cô ấy là ca sĩ. Cậu gặp cô ấy xem sao ?"

"Đó là ... nhà riêng ?"

"Vâng, nhưng cô bé rất tốt bụng. Đừng lo, cô ấy sẽ cho anh ở nhờ."

Không có vấn đề cụ thế nào khi ở nhà riêng, nhưng công việc của một "kẻ lừa đảo" thì ở nhà nghỉ tiện hơn nhiều, vì có nhiều người ở đó, và tất cả. Nhưng nếu chẳng còn nhà nghỉ, thì tôi đoán là chẳng còn lựa chọn nào khác.

"... Hiểu rồi. Tôi sẽ đi hỏi thử. Ở phía sau thị trấn, phải không ?"

"Đúng vậy. Đi thẳng đường chính từ quảng trường và cậu sẽ tìm thấy căn biệt thự. Sự thật là, lúc này là lúc nữ ca sĩ ấy đang hát. Nếu cậu bị lạc, hãy theo tiếng hát của cô ấy".

Tôi cám ơn ông già và đi.

Khi đi được vài phút, Efi lại kéo tay áo tôi và hỏi, "Bố ơi, sao mỗi khi bố nói chuyện với cư dân ở thị trấn khác nhau, bố là bố khác ?"

"Không phải bố, tao là Al. Đừng gọi tao là bố, hiểu hông ?"

"Đó, giọng bố cũng đổi luôn. Và khuôn mặt nữa. Tại sao vậy ?"

Nó đang kéo tay áo tôi. Vẻ ngoài là thứ đầu tiên mà bạn phải để ý khi bạn đang lừa người ta. Lời nói của kẻ dở hơi với cái áo không phù hợp không bao giờ đáng tin.

Tôi búng vào trán Efi một cái, nó kêu "ui" rồi ngoan ngoãn đi theo.

Tôi dắt Efi ra khỏi quãng trường. Khi tôi thấy trên đường không có nhiều người, tôi bắt đầu trả lời câu hỏi khi nãy của nó. 

"Nghe này. Con người ... Khi ta gặp người nào đó lần đầu, ta đánh giá họ qua vẻ bề ngoài. Tiếp theo là giọng điệu. Cái gì ở trong, thì, để cuối cùng."

"Nhưng giờ bố đang quàng khăn che kín mặt ..."

"Tao không rảnh để lừa người ta trừ khi nó cần thiết". Tôi kéo khăng quàng lên thêm một tí nữa.

Lâu rồi, bà buôn tin đội mũ nói khuôn mặt thật của tôi, mặt mà tôi được sinh ra, không thích hợp cho việc xã giao. Từ đó, tôi giấu cảm xúc của tôi khi nói chuyện với người khác. Chắc chắn, tôi biết rằng tôi có ánh mắt khá nham hiểm, nhưng nó có đáng sợ đến vậy ?

"Hừm, Nhưng nó có nghĩa là, nói dối, phải không Al ?" Efi hỏi với vẻ mặt bất mãn.

"Nói dối chưa được gọi là nói dối khi người ta chưa biết.". Tôi trả lời, và nó nghiên đầu như thể hiểu ít hơn khi nãy.

"Nhưng có phải nói dối là xấu ? Anh lừa gạt mọi người, nên ... anh là người xấu ?"

Trong sáng đến mức này. Và không hề có ý xấu nào.

Trẻ con rất khó để xử với những thứ này. Đặc biệt là đối với kẻ kiếm sống bằng việc lừa gạt người ta, thật khó xử.

"Al... vậy anh là người xấu hả ?", nó lo lắng hỏi lại lần nữa.

Tôi gãi đầu mấy cái, và nói Efi cái ý mà tôi vừa nghĩ ra.

"Chắc chắn, tao là người xấu. Nhưng bọn bị lừa cũng vậy."\\

\begin{center}
	******
\end{center}

% After walking straight down the path for a while, a pretty sound came into earshot, like a bird chirping from far away. No... It was a human singing

Khi đi được một lúc, một âm thanh nghe thật êm tai, như tiếng chim hót từ xa. Không ... Nó là tiếng người đang hát.

"Ồ ! Hát hay quá !"

Tâm trạng Efi tốt hơn. Mắt nó long lanh và giọng hân hoan. Tôi cảm thấy lúc mà nó im lặng và buồn bã không muốn bước đi cứ như giả tạo ấy.

"Khi ta đến căn biệt thự, thì chúng ta được nghe gần hơn. Nên, hãy đi nhanh hơn một tí."

"Vâng ạ !", Efi hét lên, và nó đi nhanh hơn. Đúng hơn thì, nó bắt đầu chạy. Chẳng lẽ nó quên hết mấy lần tôi nói là không được chạy trước sao ?

Efi càng chậm hiểu, thì chân nó càng nhanh. Tôi cho rằng nó không thể đi lạc trên con đường thẳng. Nhưng không có gì đảm bảo nó không gặp chuyện gì. Tôi rên lên và vội vàng chạy theo nó.

Túm lại, lo lắng của tôi là thừa. Khi tôi thấy Efi thì nó đã đứng trước cánh cửa màu trắng của căn biệt thự mà không gặp vấn đề gì dọc đường. Efi đang nói chuyện với một thanh niên đứng trước của. 

Anh ấy tóc xanh dương, đội cái mũ lụa được trang trí bằng dây băng vàng nghiên một bên. Hai bên đối diện cái mũ, là hai cái tai chó. Đôi tai ấy run lên mỗi khi Efi nói chuyện. Chắc hẳn anh ta có chút huyết thống của người thú.

Thực ra, tôi nhớ lại, so với các thị trấn khắc, nơi ngày nhiều người với tai thú và đuôi hơn. 

"Ồ, đó là ... Al !"

Efi chỉ tay về phía tôi và bắt đầu giải thích cho chàng trai kẻ kia. Còn tôi lo đứng thở vì cuộc rượt đuổi hồi này. Nhưng khi anh ấy nhìn tôi thì, tôi vội lấy lại vẻ biểu cảm và thò cái mặt ra khỏi khăn quàng.

"Chúc một ngày tốt lành ... Tôi xin lỗi nếu em gái tôi đến làm phiền anh. Nó chưa giở trò gì nhỉ ?"

Anh ta mỉn cười "À, đây là anh trai của em ? Tôi lo rằng em bị lạc. Thật may em chưa đi quá xa. Này anh bạn, anh đến đây có việc gì ?", anh ta hỏi bằng giọng tử tế.

"À, tôi nên tự giới thiệu. Tôi là Al, và cô gái bé bỏng này là Efina. Chúng tôi đang đi du hành. Tôi cần tìm chỗ trọ. Một cư dân thân thiện khuyên chúng tôi đến đây xin ở nhờ."

"Vâng, tôi hiểu. Vâng, Mischa sẽ chắc chắn hoan nghênh hai vị. Còn tôi, tôi là Phil, người quản gia khiêm tốn của Mischa", chàng trai trẻ Phil trả lời và cuối chào một cách lịch sự. 

Tôi cúi chào đáp lễ, rồi mở cửa. Cánh cửa cũ kĩ từ từ mở ra với tiếng kêu cọi kẹt.\\


% The moment the doors opened, a clear singing voice washed over me like a giant wave I’d only been feeling the ripples of.

Khi cánh cửa vừa mở, tiếng hát trong trẻo ùa qua tôi khiến tôi cảm giác như vừa bị xé toạt. 

Ở sảnh chính, và ngay chính giữa, là một cô gái đang hát. Xa một tí là cư dân, đang đứng thành từng lớp vòng tròn xung quanh, tất cả đều tập trung nghe hát.

Có vẻ như chúng ta đến vừa đúng lúc bài hát kết thúc, và khi cô ca sĩ hát xong nốt cuối cùng, cả sảnh im lặng trong vài giây. Ngay lập tức sau đó, vỡ òa tiếng reo hò và vô tay từ kháng giả, vang vọng khắp khán phòng.

Cô ấy cảm ơn khán giả một lượt, nhưng rồi nhận ra những vị khách không quen thuộc, cô ấy đến gần chúng tôi. 

Mỗi bước đi, đế giày cô ta chạm vào nền đá hoa cương tạo nên nhịp điệu. Khi âm thanh ấy vừa dứt, cô mở miệng.

Cô khẽ nghiên đầu, đặt cả hai tay lên ngực và nói. "Thật hân hạnh khi gặp các bạn. Tôi là Mischa. Các bạn đến đây có việc gì ?"

Giây phút đó, đôi tai thỏ khẻ đung đưa và mái tóc dài nổi bật trong tâm trí tôi.

"Chị là ca sĩ cực giỏi." Efi nhảy ra trước khi tôi kịp mở miệng

Tôi chỉ không kìm được cái sự tò mò không biên giới, tôi chỉ ... có thể ôm đầu.

Mischa, nữ ca sĩ, không hề tỏ vẻ khó chịu. Cô khụy gối xuống thấp vừa tầm mắt của Efi.

"Hehehe. Cảm ơn em. Chị thích hát. Chị rất vui khi nghe em khen vậy !". Mischa mỉn cười, và Efi cũng vậy. Một không khí yên bình lan tỏa. 

... Không ổn. Hai người họ chắc sẽ kéo mình xuống. Tôi hắng giọng, cố gây sự chú ý của Mischa.

"Xin lỗi ? Chúng tôi định ở lại thị trấn này một thời gian, và chúng tôi đang tìm chỗ trọ. Nếu bạn không phiền, có thể cho chúng tôi ở nhờ tại đây. Chúng tôi tự thu xếp thức ăn và quần áo ..."

"Ôi, trường hợp này, các bạn nên ở lại với chúng tôi. Chúng tôi cung cấp bữa ăn và tất cả những gì các bạn cần, và không tính tiền. Chúng tôi còn khá nhiều phòng ... Ồ, vâng, Phil. Hãy đi tìm một phòng trống trên lầu."

Mischa yêu cầu Phil, đang đứng sau chúng tôi. Phil trả lồi "Tôi hiểu" và bước lên như đang lướt. 

"Tôi sẽ dẫn các bạn lên phòng. Lối này." Phil chỉ tay ra phía trước và bắt đầu đi.

Tôi bảo Efi "Chúng ta đi nào", nhưng nó chẳng nhút nhít tí nào khi đứng cạnh Mischa. Nên tôi nắm áo nó và kéo nó đi.

Efi dỗi, và vãy tay chào Mischa.\\

% Going upstairs and down a long hallway, Phil, walking a few steps ahead of us, came to a sudden stop.

Chúng tôi đi lên cầu thang và đi dọc theo dãy hành lang. Phil, đi trước chúng tôi vài bước, bỗng dưng dừng lại.

Anh lấy một chùm chìa khóa từ trong túi, và chọt vào cái lỗ khóa trên cửa. Cái khóa kêu tạch và mở, anh bắt đầu nói. 

"Hãy dùng phòng này. Căn biệt thự khá rộng, và khách mới thường bị lạc ... Nhưng tôi tin căn phòng này không có vấn đề gì, vì nó ở ngay chỗ cầu thang"

Phil lặng mở cửa và mời chúng tôi vào.

Căn phòng hai giường rộng rãi, nắng sáng ùa vào cửa sổ. Trông còn cao cấp hơn cả khách sạn.

Chắc chúng tôi gặp may, khi họ cho chúng tôi ở đây, miễn phí.

"Wow, cái giường êm thật !!!"

Khi tôi bận suy nghĩ, Efi nhảy lên giường và nhúm như điên.

Á, tôi bất cẩn và quên để ý nó ... Tôi xin lỗi Phil đang cười thầm, và túm lấy Efi.

"Ồ không sao đâu. Năng động mới tốt."

Sau khi Efi dịu lại, Phil hỏi với giọng điệu đột ngột nghiêm túc. 

"... À, nhân tiện, anh định ở đây bao lâu ?"

 Bình thường, hiển nhiên ai cho mượn phòng cũng hỏi vậy. Nhưng tôi cảm giác không khí xung quanh Phil có vẻ khác.
 
 ... Hay có uẩn khúc gì ở đây ?
 
 "Ồ, để xem... Chúng tôi cần giải quyết vài thứ ở thị trấn này. Chúng tôi muốn ở lại khoảng 3 ngày. Có vấn đề gì không ?"
 
 Phil bỗng cứng đờ khi nghe nói "3 ngày"
 
 ... vậy nên chuyện này chắc chắn "có vấn đề".
 
 Tôi quan sát nét mặt Phil và đợi anh ấy trả lời.
 
 "... Không, không thành vấn đề. Đơn giản là, chỉ là có khá đông người trong căn biệt thự này. Họ đến nghe Mischa hát hoặc chỉ ghé thăm. Thường buổi trưa là đông nhất ...". Phil dừng, rồi nói tiếp "Nó có ổn với anh không ?", anh ta hỏi với vẻ mặt căn thẳng.
 
 Phil cố gắn giấu ý nghĩ của mình một cách tuyệt vọng, nhưng biểu cảm thì càng lúc càng không tự nhiên. Tôi cho rằng những gì anh ta nói cứ như là trả lời chất vấn trước công chúng.
 
 Ngược lại, tôi trả lời với mặt rạng rỡ. "Không, không sao ! Đừng lo. Như anh thấy đó, em gái tôi rất ồn ào, nên tôi quen rồi."
 
 Nghe vậy, đôi vai của Phil thả lỏng, và mỉn cười. Còn Efi, như muốn đưa ra ví dụ, vùng vẫy tay và chân khi tôi đang giữ.\\
 
 
 % “...Alright, let’s make it simple. You’re Efi, my little sister. That’s all the info you need to remember. Don’t talk any more than you need to. Got it?
 
 "... Được rồi, nói ngắn gọn. Mày là Efi, em gái bé nhỏ của tao. Đó là những gì mày cần biết. Đừng nói gì nhiều hơn những gì mày cần nói. Hiểu không ?"
 
 Một khi Phil đi khỏi, tôi cho Efi một vài lời dặn dò để nó không gây rắc rối. Tôi từng cho nó dùng biệt danh, nhưng nó cứ quên và tự gọi mình là Efi, nên khỏi.
 
 "Hừm,... vậy em không được nói gì hết à?". Nó nhíu mày và nghiên đầu hỏi.
 
 "Ví dụ như quá khứ. Mày đã đi đâu, thấy gì, nếu như được hỏi như vậy thì, đừng nói gì cả. Mấy người nói dối dỡ thường nói những thứ mà họ chẳng bao giờ được hỏi, và thậm chí không biết câu chuyện họ bịa ra nghe không đồng nhất."
 
 Efi cực kì bối rối khi nghe tôi giải thích, và nó xoay xoay cái đầu.
 
 "Vậy em nói gì nếu em được hỏi ?"
 
 "Đơn giản, cứ trả lời "Em là tên ngốc, nên em không biết!""
 
 "Tên ngốc là gì ?"
 
 ... Phải chăng nó là tên đại ngốc ? Phải, thật tiện cho tôi.
 
 "Có nghĩa là mày có não phẳng."
 
 "Hừm, có não phẳng là xấu ?"
 
 "Có nghĩa là, mày sẽ gặp rất nhiều rắc rối phía trước. Chắc chắn luôn."

Tôi chẳng giỏi chăm sóc trẻ con. Nhất là khi chúng nó còn trong sáng và ngay thẳng ... và cứ tỏ vẻ ngây thơ đâm đầu vào rắc rối với cái đầu rỗng, như oắt con này.
 
 "Hừm...."
 
 Ngay cả khi tôi giải thích đến mức này, nó cũng có vẻ chưa hiểu. Cứ thế này sẽ thật tốn thời gian nếu tôi tiếp tục giải thích.
 
 "... Mọi chuyện không tệ lắm đâu. Nên mày đừng lo."
 
 "Vâng ạ, em hiểu rồi."
 
 Tôi tự hỏi nó có thực sự hiểu ? ... Nó sẽ hiểu, từ từ cũng hiểu. Đến khi Efi hiểu "tên ngốc" là gì, tôi tự lo được. Tôi chẳng phiền.
 
 "Được rồi ... Vẫn còn thời gian trước khi trời tối, nên tao ra thị trấn một chút. Nếu tao muốn tìm thấy con chim lam, tao cần biết trước một vài thứ."
 
 "Em cũng vậy !" Efi nhảy ra khỏi giường khi tôi vừa đứng dậy. Tôi mém nữa đụng đầu nó. Tôi giữ thăng bằng, hất tay nó ra khỏi cổ tay tôi.
 
 "Tao tưởng mày không muốn nói đi bộ nữa chứ ? Thị trấn này to lắm, nên quên nó đi. Ở yên đây."
 
 Tôi ném nó lên giường, và nó cứ la ó lên.
 
 "Tao sẽ mua một ít bánh kẹo trên đường về."
 
 "Efi sẽ ở yên đây !!!" Nó nhìn với ánh mắt lấp lánh.
 
 Tôi liếc mắt nhìn nó, rồi ra khỏi phòng, sau đó ra ngoài biệt thự.\\

 \begin{center}
 	******
 \end{center}
 
 % A few hours had passed since we arrived in town, but the bustle outside showed no sign of dying down. I headed for the fountain plaza and found the middle-aged man who told me about the mansion, taking a break on the same bench as before.
  
 Đã vài giờ trôi qua kể từ khi chúng tôi vào thì trấn, nhưng vẻ ồn ào, náo nhiệt có vẻ như chẳng bao giờ chấm dứt. Tôi đi về phía quảng trường, và gặp lại ông già trung niên đang ngồi nghỉ trên cái ghế, đúng cái chỗ ổng ngồi khi nãy.
 
 Thấy tôi, ông ngẩn mặt lên, "Ồ, người hành lữ ! Cậu đã tìm ra chỗ trọ chưa ?"
  
 "Vâng, chúng tôi không gặp vấn đề gì. Chúng tôi đã đến căn biệt thự ông bảo, và cô Mischa đã chấp nhận cho chúng tôi ở nhờ. Ông thật là có ích."
 
 "À, được rồi, được rồi !" Ông già mỉn cười, đập tay vào lưng tôi vài phát.
 
 Tôi có thể nói rằng, có rất nhiều người tốt quanh đây, như ông già này. Như vậy, mọi thứ sẽ dễ dàng hơn nhiều.
 
 Ông già coi đồng hồ. "Ồ, đến giờ rồi ! Tôi về đây, hoặc vợ tôi sẽ nổi giận... Ồ, đừng nói là tôi đã nói vậy cho ai hết nha. Chúc một ngày tốt lành !". Chào tạm biệt xong, ông già vội vã đi.
 
 Tôi quay lại và tiếp tục chuyến đi. Tôi đang đi thì thấy vài quý bà tụ tập dưới bóng cây thì thầm về cái gì đó.
 
 Tôi đi ngang, giả vờ ngây thơ, và nghe trộm. Cuộc đối thoại bỗng làm tôi chú ý. "Mọi chuyện có ổn không nếu ta cho nó ở nhờ trong biệt thự ?" "Có nên đuổi họ đi không ?"
 
 Có phải mấy quý bà ấy đang nói về chúng ta ? Ngay cả khi tôi đi ngang qua, tôi cảm giác là họ đang nhìn tôi, nên, chắc là vậy rồi.
 
 Ngay khi tôi vừa đến đây, tôi cảm giác có cái gì đó không ổn. Tôi không chỉ ra chắc chắn nó là cái gì, nhưng nó làm tôi thấy bất an.
 
 Nhưng mà, đoạn đối thoại đó chẳng liên quan trực tiếp gì đến cuộc tìm kiếm chim lam của tôi. Vậy nên, quên nó đi.
 
 Tôi lắc đầu để xua tan những lo lắng ấy, và quay lại vấn đề chính, tìm con chim lam.
 
 Tôi quấn lại khăn quàng khi xem hoàng hôn.
 
 "Hy vọng vụ này êm xuôi."
 
 Tôi để vụt ra những từ ngữ ấy một cách vô thức, và vội thu lại. Đừng nói gỡ như thế, lỡ có chuyện gì thì sao ? ... Nhưng đã trể quá rồi.
 
 Tôi quyết định trở về nghỉ.
 
 Tôi chợt nhớ ra, tôi đã hứa mua bánh cho bé rồng. Tôi đi tìm mua bánh với bước chân nặng nè.\\
 
  \begin{center}
 	******
 \end{center}
 
 Tôi trở về căn biệt thự, mở cửa trước và thấy Mischa đang hát trong sảnh chính. Nhưng khác lần trước, không có cả khán phòng, mà chỉ có một thính giả, đó là Efi.
 
 "Ồ, chào mừng trở về ! Chắc anh đã mệt."
 
 Mischa để thấy tôi và nở nụ cười mềm mại. Chẳng ai quan tâm tôi, những cực nhọc, mệt mỏi mà tôi đã trải qua. Tôi thực sự cần.
 
 "Hở, Al, anh đã về rồi ! Bánh, bánh đâu ?"
 
 ... Nó là gì lý do tại sao.
 
 Tôi không muốn cho Mischa thấy vẻ mệt nhọc ấy, nên tôi chuẩn bị mặt mũi, gỡ cái khăn quàng cổ ra, và lấy ra hộp bánh mà tôi đã mua ở tiệm gần đây. Mắt Efi rạng rỡ ngay lập tức, và nó cố giật hộp bánh ra khỏi tay tôi, nhưng tôi vẫn giữ chặt cái hộp.
 
 Khi Efi ăn đồ ngọt, giống như nó gỡ một vài giới hạn, và đôi khi, nó phun lửa. Có vẻ như một vài con rồng có thể kiểm soát nguyên tố tự nhiên, ngoài năng lực đặc biệt
 của nó ra. Tôi tự hỏi liệu Efi có phải một trong số những con rồng ấy không ? Tôi đã phải bỏ đi cả đống quần áo cháy.
 
 "Này, mày không ăn ở đây được. Phải về phòng mà ăn chứ."
 
 Con người thật của tôi mém trỗi dậy trước mặt Efi, nhưng tôi phải chú ý giữ gìn hình ảnh anh trai tốt bụng trước mặt Mischa.
 
 "Sao vậyyyy ? Em muốn ăn với chị ca sĩ." Efi từ chối và kéo lấy hộp bánh.
 
 Thường thì nó sẽ phàn nàn, nhưng rồi nghe lời, nhưng tại sao nó lại chống đối ngay lúc này ? Tôi đoán là tôi vẫn chưa biết cách nuôi nó.
 
 "Hehehe, em đừng để ý chị. Em cần phải về phòng sớm. Hãy về phòng và ăn bánh, Efi mê đồ ngọt nè !!"
 
 Mischa nói với nụ cười hóp hồn, rồi cúi chào và đi lên cầu thang.
 
 ... Thật bất ngờ.
 
 Tôi không nghĩ cuộc chiến giành gói bánh được diễn rất tốt. Mischa ấy chẳng mảy may nghi ngờ gì. Có thể thậm chí, cô ấy còn thấy chúng tôi như anh em thân thiết.
 
 "Này, ta cùng ăn bánh nào !!!"
 
 Tiếng Efi đưa tôi về thực tại, và tôi lại cảm thấy cái mệt mỏi ùa về. Ây da... Tôi phải chịu đựng cảnh này bao lâu nữa ?
 
 Tôi có nên hỏi ai đó về loài rồng này ? Số rồng tôi biết chỉ đếm được trên đầu ngón tay, của một tay... Tôi cảm thấy ý này không ổn. Nên tôi dừng lại.
 
 Hay là đi ngủ đi. Hay, ý hay đó.
 
 Tôi vội leo lên cầu thang mà chẳng buồn lắng nghe Efi rên rỉ. Nó kêu "Ah!" và vội theo sau tôi.\\
 
 
 Khi chúng tôi về đến phòng, tôi cởi áo khoát và ngã lên giường ngay lập tức. Hiếm khi tôi thấy mệt thế này.
 
 Tôi thường phải làm việc với nhiều người "đặc biệt", và tôi tự hào nói rằng, tôi quen với điều này. Nhưng đây là bé rồng. Vừa là "bé", vừa là "rồng". Tôi cũng chẳng thể nào hiểu cách nó suy nghĩ như thế nào.
 
 Tôi cảm giác như bị cái mệt mỏi nghiền nát. Trong lúc đó, Efi nhìn tôi lo lắng. Tôi quay lại nhìn vào mặt nó. 
 
 Tôi nhìn vào mặt nó thêm lần nữa. Tôi nghĩ, nó chẳng khác người bình thường gì cả. Tôi tự hỏi tại sao rồng lại bắt chước hình dạng con người.
 
 Tôi tự hỏi Efi đang nghĩ gì khi nó nhìn tôi nãy giờ. Có lẽ nó lo cho tôi ? Vậy thì, ước gì nó để tôi yên.
 
 Tôi chẳng còn sức nói gì cả, chỉ nhìn. Và Efi, như có gì đó vừa lóe lên trong đầu nó, nó mỉn cười hớn hở và hỏi một cách phấn khởi.
 
 "Chúng ta ngủ chung nhé, được không ?"
 
 "Để tao yên", tôi thẳng thừng từ chối
 
 
 
 













