% Source http://vgperson.tumblr.com/post/137776404835/lieat-the-novel-part-ii
%   \section{The Lie-Eating Dragon and the Forgotten-Color Songstress}
%	\subsection*{Ngày 5 ???}

Tớ ghét chính mình.

Tớ chậm chạp, và chẳng có gì về tớ đáng để khen. Tớ không có hy vọng về tương lai. Tớ chẳng có gì, tớ vẫn khỏe.

...Bởi vì, tớ chẳng lo gì cả. \\

Nhưng từ khi tớ gặp cậu, tớ cảm thấy gì đó thay đổi. Tớ chẳng có thêm cái gì mà nhìn thấy được, nhưng tớ cảm thấy cậu đã cho tớ gì đó. Lúc trước tớ còn chẳng mảy may nghi nhờ mình là kẻ trắng tay, tớ cảm thấy xấu hổ.

Ngày tớ gặp cậu, cậu cứu tớ bằng giọng hát ấy, giọng hát vui vẻ hạnh phúc từ cái băng ghế trong công viên. Tớ nhớ rất kĩ giây phút đó.

Nên tớ nghĩ: "Tớ muốn trở thành người có thể cho cậu gì đó."

Chắc cậu đã quên rồi, nhưng tớ không để bụng đâu \\ 

% Really

"Thật à ?", chim lam đứng cạnh tôi, hỏi.

Đó là một đêm thanh tịnh, chỉ có tiếng cây xào xạc. Tôi ra khỏi biệt thự và trò chuyện cùng chim lam.

"Em không thấy vậy."

Nó không thấy vậy. Tại sao vậy ? Tôi không biết nữa.

"Anh đã nghĩ đi nghĩ lại kĩ rồi. Tốt hơn hết, mọi thứ cứ như thế này. "

"Hở ?" Nó gật đầu, đôi cánh đung đưa trong gió lạnh. "Anh, Phil, anh nghĩ mọi thứ tốt nhất là theo cách này hả ? Anh thực sự nghĩ vậy hả ?"

"Ý em là sao ?"

"Hạnh phúc không phải là thứ lâu dài." chim lam lẩm bẩm một mình, nhìn có vẻ đau khổ.

"Em đau lắm hả ?"

"Em hả ? Vâng, khi em nhìn tụi anh, em rất đâu lòng."

"Anh chẳng sao cả"

"Không, em nói là "tụi anh"."

Vậy, không chỉ mình tôi ?

"Còn ai ngoài anh ra nữa ?"

"Ý em là ... hmm, em không nói đâu. Anh sẽ biết sớm thôi."

Nó trả lời không rõ ràng mà lại còn thâm nữa. Tôi cảm thấy bối rối.

Rồi, chim lam đột ngột ho dữ dội. Tôi ngạc nhiên nhìn nó. Nó từ từ ngẩn đầu và mỉn cười. Ngực tôi đau quặn.

"... Có vẻ như em là người chịu đau đớn nhiều nhất."

"Em ổn. Em được sinh ra để làm việc này. Chẳng sao nếu em làm vì anh, Phil."

Tại sao chim lam lại kiệt quệ vì tôi ? Cứ như vậy từ khi chúng tôi gặp nhau.

Chim lam trong truyền thuyết mang hạnh phúc đến không chỉ một, mà rất nhiều cư dân.

Bây giờ, cư dân trong thị trấn này có vẻ hạnh phúc, nhưng cũng ngần ấy người bất hạnh. Chắc rằng, trước khi tôi gặp chim lam thì mọi việc không thế này.

"Anh không nghĩ là để làm ai đó hạnh phúc, một ai đó phải bất hạnh... Không phải vậy chứ ? "

Tôi lo lắng nhìn chim lam. Nó hít sâu thật sâu để đưa nhịp thở trở lại bình thường.

"Năng lực của em không cần phải thế. Chỉ là.. em đoán là anh có thể nói năng lực này cho cả hạnh phúc và bất hạnh."

"Ý em là sao ?"

"Nếu anh biết hạnh phúc thực sự là ... "

Chim lam nhìn một cái gì đó xa xôi. Nó có ý gì vậy ? 

"Giờ thì ... Em đi ngủ. Phil, anh cần phải gọi Mischa dậy sớm nữa đó. Anh nên ngủ đi."

"À, phải ha"

Hiếm khi tôi gặp chim lam vào tối thế này.

Ngày đầu tôi gặp chim lam, nó nói tôi là "Cú đêm", nên khi tôi muốn nói chuyện, tôi gặp nó vào ban đêm. Nhưng sau đó, nó nói "gặp nhau ban ngày cũng không sao," và thế là buổi gặp mặt ban đêm thưa dần. 

Đêm ấy, tôi muốn hóng mát, nên đi vào vườn. Và tôi gặp chim lam đang đậu trên hàng rào. Và chúng tôi nói chuyện.

Sau khi tạm biệt chim làm, tôi nhấc mình lên khỏi bức tường gạch xung quanh vườn hoa. \\

"Nếu tôi biết hạnh phúc là gì ..."

Những từ ngữ ấy cứ xoáy trong đầu tôi đến khi tôi về phòng.

Chuyện gì sẽ xảy ra ?

Tôi có thể giữ hạnh phúc thế này ? Hoặc tôi sẽ bất hạnh ?

... Tôi tự hỏi nếu chim lam thực sự biết hạnh phúc là gì.

Lời nói của chim lam làm tôi trằn trọc cả đêm ấy.
 


