% Source http://vgperson.tumblr.com/post/137776404835/lieat-the-novel-part-ii
%   \section{The Lie-Eating Dragon and the Forgotten-Color Songstress}
%	\subsection*{Ngày 5}

Hôm nay tôi không ngủ quên nữa, và dậy đúng giờ, Chắc là do hôm qua ngủ sớm.

Tôi bảo Efi tìm chim lam lần nữa trong biệt thự. Còn tôi thì đi lanh quanh trong thị trấn.

Tôi từng nghe chim lam xuất hiện trước cư dân, bao gồm cả cư dân buồn bã. Từ khi mọi người thấy căn biệt thự quá rùng rợn để lại gần, thì những cuộc gặp gỡ phải diễn ra bên ngoài.

Vậy nên tôi cần phải tìm ngoài kia luôn, và có khả năng gặp lại nó. ... Nhưng khả năng chẳng là gì cả. Nếu cho rằng "chỉ một lần mỗi tuần", thì cuộc điều tra hôm nay cơ bản là vô ích.

"Cảm giác như vẫn dậm chân tại chỗ." 

Đúng là, tôi chưa tận mắt gặp chim lam, nên cứ cảm giác như không có tiến độ gì cả. Có lẽ tốt hơn nếu cùng Efi tìm trong biệt thự.

Bỗng nhiên, tôi thấy cái bóng thật to trên đầu.

"Hừm ?"

Có gì đó đang bay. Không phải chim.

Trông như một người với cái cánh trên lưng. Nhưng không phải chim lam. Cái mặt quen quen, tôi chắc chắn không nhầm được.

Khi tôi suy nghĩ, gã đó nhận ra tôi, và dần dần hạ cánh.

Không, ổn cả. Anh hãy ở trên đó.

"... Cả tháng rồi nhỉ."

Dù tôi nói vậy, hắn cũng nhẹ nhàng đáp xuống trước mặt tôi và nói.

Hắn ta có mái tóc xanh lẫn trắng, và có hai kim cài tóc khác màu. Hắn đeo một cái bịt mắt màu đen. Cơ thể cao to. Hắn ta là rồng, đội úy cảnh sát, tên Neil.

"Điều gì đem đại úy đến đây ? Ngắm cảnh ?"

"Không, " Neil đáp, cười cay đắng. "Anh thấy đó, tôi đang đi làm nhiệm vụ."

Tôi và đại úy quen nhau từ nhỏ. Tổ đội của anh ta chuyên điều tra và giải quyết các vụ án khắp nơi trong nước. Còn tôi có thói quen là thường dính vào mấy vụ án đó. Vậy nên chúng tôi thường gặp nhau.

"Hôm nay anh dùng tên gì ?"

"Al"

Và đại úy, tất nhiên, dễ dàng hiểu tôi. Nên chẳng có lý do để che đậy thái độ hàng ngày.

"Al, tôi thấy. Nếu anh ở đây, nghĩa là có bằng chứng về cái gì đó đang diễn ra trong thị trấn này."

"Đại úy, thật là đắng lòng..."

Và lại lần nữa, tôi cảm thấy vậy. Cảnh sát được gửi đến đây cho tôi biết, có gì đó đang diễn ra.

Nhưng liệu đây có phải do con người gây ra ? Sẽ tốt hơn nếu tôi biết thêm được gì đó từ đại úy.

"Hở ? Oa !"

Bỗng dưng, một đám sương mù đen xuất hiện dưới chân tôi, và tôi bước lùi thật nhanh. Cảm thấy nguy hiểm, tôi móc con dao giấu trong cổ tay và chĩa vào đám sương đó.

"À, xin lỗi, có vẻ làm anh sợ. Cái bóng này là cấp dưới của tôi. Nên cất hàng đi." Neil khiển trách.

Cái bóng ? Cấp dưới ? Tôi đoán là cấp dưới của Neil đều là rồng cả.

Chắc chắn là cái bóng này là một sản phẩm của rồng. 

Đám sương đen ấy tụ lại một khối. Rồi từ khối đó trồi lên một chàng trai trẻ tóc hồng và ánh nhìn sắc lẹm. Anh ta trẻ măng, làm tôi thắc mắc thế nào mà nó trở thành cấp dưới của Neil được. Nhưng dòm bộ đồng phục thì đúng thiệt.

"Đại úy, tôi đã điều tra, và tin rằng mọi chuyện giống như anh nghĩ."

"Được rồi, làm tốt lắm".

Neil khen tin thần làm việc của chàng trai trẻ này, và anh ta thấy và nhìn tôi. Đúng hơn là, nhìn chằm chằm.

"Gì vậy ? Sao anh lại nhìn chằm chằm vào tôi ngay lần gặp đầu tiên ?" Gã này không có kĩ năng xã hội.

"Đại úy, người này ... là ai thể ?"

Nó lờ tôi và hỏi Neil. Hả, gã này không ưa tôi ? Chẳng hiểu tại sao.

"Anh ấy là Al. Và Al, đây là Brett. Anh ta chỉ là người học việc, nhưng rất tài giỏi." Neil vỗ vai Brett và chỉ vào tôi.

Dù đã được giới thiệu, Brett vẫn tiếp tục nhìn chằm chằm tôi, và không hề mở miệng.

"... Xin lỗi. Anh ta khá nhút nhát." Neil xin lỗi và cười vặt vẹo.

Tôi cảm giác "nhút nhát" này có gì đó ngoài giới hạn... Hơn nữa, gã Brett này trông quá trẻ để làm cảnh sát.

"Có phải đội đang làm nghĩa vụ trông coi thằng nhãi này ?"

"Sao mày dám nói như thế với đại úy ? Tao sẽ kéo mày ra khỏi thành phố và bẻ từng cái tay"

Đó là thứ đầu tiên mà nó nói với tôi à ? Chắc chắn không phải thứ các bạn muốn nghe từ những người dâng hiến cả cuộc đời để bảo vệ hòa bình.

"Brett, cậu thật thô lỗ. Xin lỗi nếu cậu ta làm phiền cậu, Al. Cậu ta chưa thể tự kiểm soát năng lực của mình, nên đừng làm cậu ấy kích thích."

"Này, tôi chưa làm gì cả nhé."

"Tôi ngửi thấy tội phạm trong người này"

Nói một cách nghiêm túc. Đúng thế, một phán quyết tại chỗ.

"Có thể trực giác cậu đúng, nhân cậu cần chứng cứ để buộc tội. Trước đó, anh ta vẫn được xem là một công dân bình thường để chúng ta bảo vệ."

... Đại úy đang nói đỡ cho tôi ? Cảm giác như đúng là thế.

"Cái đống sương mù đen này để làm gì ?" Tôi đổi chủ đề. Brett cũng không nói chuyện với tôi, nên Neil trả lời thay cho hắn.

"Cậu ta là rồng bóng đêm. Như anh thấy, cậu ta có thể biến thành dạng bóng và đi xuyên qua vật rắn. Và năng lực này không chỉ dùng được mỗi trên bản thân mình, mà còn vật khác nữa."

"Năng lực hữu dụng thật."

Tôi cố khen một câu tình cờ, nhưng Brett không có phản ứng gì. Tôi bắt đầu thấy ngượng.

"Này đại úy, anh đến thị trấn này làm gì ?", Tôi hỏi, vì đây là thứ đầu tiên trong đầu tôi. "Tôi đã đến đây được vài ngày, nhưng vụ này có vẻ nghiêm trọng nếu anh đã trực tiếp đến đây"

Cả đất nước này, mỗi ngày có cả khối vụ, to và nhỏ. Lực lượng cảnh sát có rất nhiều thành viên nên họ có thể tách ra và phân công cho mọi người tất cả các vụ án.

Trong đó, phần lớn các vụ giết người và tranh chấp, đại úy Neil ra tận nơi. Hoặc là ... vụ án có liên quan đến rồng. 

"Phải ... Chúng tôi nhận đươc yêu cầu của ai đó trong thị trấn. Rất nhiều cư dân nơi này gặp rắc rối bởi hiện tượng huyền bí."

"Hiện tượng huyền bí ?"

"Thấy người đã chết, thấy những thứ không nên có ở đó. Thậm chí bị bọ khổng lồ tấn công."

... Thật là tàn bạo.

Cư dân buồn bã đã nói tôi là nhìn thấy ma và những thứ ấy, nhưng hiện tượng có vẻ như không thực tế. Tôi bắt đầu lo liệu có loại thuốc đáng ngờ nào đó đang lan rộng.

"Những hiện tượng đó đưa anh đến đên, phải không ?"

"Đúng. Chúng tôi tin rằng có một thế lực đã gây ra nhưng hiện tượng kì lạ đó. Thế lực này đang bao phủ cả thị trấn. Cư dân ở đây không biết nguồn gốc và mục đích cuối cùng của thế lực này. Chúng tôi không thể phủ nhận thế lực này có liên quan tới tổ chức tội phạm hay khủng bố, nên chúng tôi quyết định tới đây."

Một thế lực che phủ cả thị trấn ?

Thị trấn này to hơn các thị trấn khác, và dân số đông hơn. Nếu thực sự có thế lực gì đó đang ẩn mình, gây ra những hiện tượng kì là ... Có lẽ cư dân buồn bã, và niềm tin rằng những cư dân khác là quái vật, cũng vì ảnh hưởng của thế lực này.

Neil muốn nói tiếp, nhưng Brett chen ngang một cách không hài lòng. "Đại úy, anh có nên cho gã này quá nhiều thông tin của ta không ?"

"Có vấn đề gì sao ? Những gì tôi nói là để đề phòng."

"Anh đã nói quá nhiều chi tiết. Hắn ta đủ đề phòng rồi. Thời gian là vàng bạc. Ta quay lại điều tra nào."

Mặc dù là người học việc, thằng nhãi nói như cùng đẳng cấp với Neil. Lần nữa, năng lực và tuổi của rồng không biểu hiện qua vẻ bề ngoài của chúng. Có thể, bọn họ đồng trang lứa.

"Xin lỗi, chúng tôi sẽ quay lại nhiệm vụ. Hãy tránh xa những cư dân từng thấy hiện tượng lạ, và đừng can thiệp."

"Hiểu rồi, còn một thứ nữa tôi cần hỏi ..."

"Đại úy !"

Tôi giữ Neil lại để hỏi thêm, nhưng Brett hét gọi Neil. Bóng đen bao quanh nó như ngọn lửa. Chắc nó đang cấu.

"Đừng lo, nhanh thôi."

Nếu tôi có thể hỏi đại úy việc này, thì tôi đủ dữ kiện để xếp các mảnh ghép lại với nhau. Tôi muốn hỏi ngay tại đây.

"Gì vậy ?", Neil trả lời. Brett nhìn chua cay và lùi vài bước.

A ha. Tôi có thể moi tin từ Neil một cách nhanh chóng, nhưng gã Brett này khá cứng đầu. Gã đang cố tỏ ra mình là đối thủ đáng gờm.

"Về thể lực đang bao trùm thị trấn này..." \\

% I see. Thank you. I just wanted to know

"... Tôi hiểu. Cảm ơn. Tôi chỉ muốn biết."

Tôi quay lại và quyết định rời đi. Vì tôi cảm giác độ khó chịu của Brett sắp vượt quá giới hạn.

"Tại sao chú mày khó chịu vậy ?"

Tôi bỗng nghĩ ra, và quay người lại hỏi.

Nó vẫn cứng đầu, không chịu nói, nên tôi hỏi lại. "Ta chưa từng gặp nhau. Vậy có lý do gì ? Nấu tao không biết tại sao tao làm mày khó chịu, thì tao chẳng giúp được gì."

Có lẽ tôi vừa ném que diêm vào dầu lửa. Nhưng trái lại, Brett từ tốn trả lời khi nó xoay mặt chỗ khác.

"... Tôi cũng không biết. Nhưng nhìn thấy vết sẹo trên cằm anh, tôi bỗng thấy khó chịu. Chỉ vậy thôi."

... Một lý do phi lý thật.

Tôi đã kéo khăn quàng xuống, để nói chuyện với Neil, nên vết sẹo lộ ra. Nhưng sẹo này tôi có lâu rồi, và chẳng liên quan gì đến Brett.

Neil nhìn chúng tôi im lặng, rồi lẩm bẩm.

"Anh bị sẹo đó khi nào ? Từ lúc trong tổ chức à ?"

"Ờ đúng. Sau khi tôi vui vẻ nói lời chào tạm bệt."

"Hừm ...". Neil suy tư. "Rồng thường không chia sẽ kí ức hoặc kỉ niệm với chủ của nó. Nhưng có một vài trường hợp hiếm. Có thể Brett là một trong số đó."

"Ý anh là sao ?", Brett hỏi một cách tò mò.

"Anh thấy đó, anh ta từng là ..."

"Tôi đi. Anh sắp đẩy tôi vào tâm trạng không tốt."

"À", Neil đáp.

Việc tôi từng trong tổ chức ấy đã qua rồi. Nhưng mọi thứ vẫn chưa chấm dứt. Tôi không quan tâm quá khứ, vì tôi có thói quen nhớ những thứ mà tôi không muốn nhớ.

... Nhưng, hở. Rồng bị ảnh hưởng bởi chủ của nó không phổ biết ? Nếu bạn so sánh tôi và Efi, thì, bạn hiểu rồi đó. Nó không giống như trẻ con "người" có cái mặt của bố mẹ nó.

"Vậy thì, tôi đi đây. Gặp lại sau nhé."

Tôi quyết định chẳng còn thông tin gì để moi và chuồn lẹ.

Neil chào tạm biệt, còn Brett chẳng nói gì, và vẫn nhăn nhó. Sau chốc lát, tôi nghe tiếng như "tạch" \\

\begin{center}
	******
\end{center}


 % Oh
 
 "Ồ !"
 
 Mình nghe giọng hát. Nó yên tĩnh, nhưng hay thật.
 
 "Có phải chị ca sĩ ?"
 
 Mình nghĩ mình vừa gặp chị ở cổng chính, chị ấy vừa đi nơi khác à ? Tiếng hát phát ra từ sau vườn.
 
 "Chị ca sĩ ! Hả ?"
 
 Mình ra vườn, nhưng không phải chị ca sĩ. Thay vào đó, là chim lam, đang ngồi hát trên hàng rào. 
 
 "Ê, Efi, chúc một ngày tốt lành."
 
 Chị ấy nhận ra mình và nhảy xuống. Rồi, không còn nghe tiếng hát nữa. Mình đoán là chỉ mình chị ấy hát ?
 
 "Chị đang hát hả chị chim lam ?"
 
 "Đúng rồi, Mischa dạy chị hát."
 
 Mình biết mà, đây là bài hát của chị ca sĩ.
 
 "Chị là bạn cũ của chị ca sĩ hả ?"
 
 "Hừm, Ý chị là, chị đã ở với Mischa từ lúc mới sinh... nhưng chỉ mới một năm và một vài thay đổi. Nên cũng chưa phải là bạn "cũ" nhỉ. "
 
 Một năm và vài thay đổi ? Mình quan sát cơ thể chị chim lam. Chị ấy lùn hơn mình một tẹo. Người ta lớn nhanh thế này chỉ trong một năm à ?
 
 Và rồi, chị ấy là chim. Và Al nói có khá nhiều loại "người".
 
 "Chị biết Mischa và Phil rất rõ. Bao gồm cả quá khứ của họ. Tò mò không ?"
 
 "Chị biết thật à ? Vâng, em tò mò lắm."
 
 "Và rồi, em vừa thất hứa với chị, nên chị nghĩ..."
 
 Ối. Phải, chị ấy bảo mình giữ bí mật. Nhưng mình kể Al mất rồi.
 
 "Xin lỗi, chị chim lam...", mình lí nhí một cách xấu hổ. Mình thất hứa. Mình là trẻ hư.
 
 "À, chẳng sao cả. Chị nói vậy như kiểu chào hỏi. Chẳng quan trọng gì."
 
 "Thật à ?"
 
 "Thật chứ !"
 
 Nhìn chị chim lam chẳng thấy vẻ khó chịu gì cả, nên mình nghĩ chị ấy không để bụng. Vậy là, mình cũng không cần để ý việc này luôn, nhỉ ?
 
 "Này... còn quá khứ của Mischa và Phil, phải không. Chị chỉ kể được những gì chị biết."
 
 "Ai kể cho chị vậy ?"
 
 "Tất nhiên là Mischa và Phil rồi."
 
 Hở ? Chị ấy được kể về quá khứ của chị ca sĩ và ngài quản gia ?
 
 "Nhưng chị ca sĩ chẳng nhớ gì mà ?"
 
 Đó là những gì ngài quản gia nói hôm qua. Chị nhớ ba ngày, rồi quên. Nên chị ấy không thể kể về quá khứ, phải không ?
 
 "Đúng, Mischa bị bệnh mất trí nhớ. Nhưng chị biết được thì Mischa không quên hết mọi thứ. Nếu cái gì đó quan trọng, chị ấy sẽ nhớ"
 
 Phải rồi. Ngài quản gia cũng nói vậy. Chị ấy nhớ tên mình, gia đình và những thứ quan trọng như vậy.
 
 "Mischa nhớ về quá khứ, chỉ những thứ mơ hồ. Như là cái ghế trong công viên mà chị ấy luôn ngồi. Đó là chỗ quan trọng. Nhưng chị ấy chẳng biết mình đã làm gì ở đó."
 
 "Chắc chị ca sĩ ngồi trên ghế và hát ?"
 
 "Ai biết," chim lam làu bàu, nhìn về xa xăm.
 
 "Mỗi sáng Mischa mất trí nhớ, em biết Phil làm gì không ? Anh gọi Mischa dậy, và nói "rất vui được gặp bạn.""
 
 "... Nhưng anh ta đâu chỉ gặp chị ca sĩ lần đầu ?"
 
 Ý mình là, Phil nhớ. Nhưng tại sao anh lại nói "rất vui được gặp bạn" ?
 
 "Bởi vì Mischa sẽ bối rối. Chị sẽ quên về Phil, nên anh không thể nói "ta gặp lại nhau" hay đại loại thế."
 
 "hừm ..."
 
 Nghe lạ thật. Chị ấy không quên tên mình, hay bài hát. Nhưng chị lại quên về người quản gia.
 
 "Sao vậy ?"
 
 "Ngay cả khi ngài quản gia đã ở với chị lâu năm ... ?"
 
 Mình cũng chỉ vừa nở ra và trải qua một vài thay đổi. Nhưng Al quan trọng đối với mình.
 
 Tại sao ngài quản gia đã ở cả đời cùng chị ấy lại chẳng quan trọng ?
 
 "Phải. Nhưng chị nghĩ nó là thứ gì đó mà ta phải chấp nhận. Nó như một lời nguyền, ta không thể chống lại nó."
 
 "Ngay cả khi chị ca sĩ là người tuyệt vời ..."
 
 "Chẳng quan trọng. Lời nguyền và ban phước được phân phát công bằng cho mọi người."
 
 Ai biết công bằng lại kinh khủng vậy... Bạn thường nói gì vào những lúc thế này ? Mình đoán là ... 
 
 "Em thấy tội nghiệp chị ca sĩ..."
 
 "Phải. Cả người bị nguyền rủa và người được ban phước. Mischa và Phil cả hai đều đáng thương. Chị chắc là chị cũng đáng thương vì chìa tay ra cho họ."
 
 "Hả ?" Chị ấy muốn nói gì vậy ?
 
 "Ha ha, chị đoán là ta không nên gọi mình đáng thương. Đó là cái mà người ta gọi mình."
 
 Và rồi chị chim lam ngáp và uốn mình.

 "Àhh, hôm nay chị mệt rồi. Chị đi nghỉ sớm."
 
 "Sao lại như vậy ?"
 
 "Hừm, chắc tại chị năng động quá. Chị không nên làm việc vất vả thế này, nhưng chị nhận ra đó là vì Phil"
 
 "Vì ngài quản gia ?"
 
 Chị chim lam dụi mắt và trả lời "Phải, Phil là người bạn quý của chị." 
 
 "Ừm... Em hiểu !"
 
 Chị chim lam đang cố hết mình vì bạn của chị ấy. Mình muốn cố gắng hơn nữa. Và mình sẽ lớn lên nữa.
 
 "Mmm ... chị không thể thức được nữa. Chúc ngủ ngon, Efi."
 
 "Chúc ngủ ngon, chị chim lam", mình vẫy chào, chị ấy vỗ cánh bay đi.
 
 Chị chim lam lảo đảo bay qua bên kia hàng rào. \\
 
 % Welcome back, Al
 
 "Al, anh về rồi !"
 
 Mình đứng đợi ở cổng chính, và khi Al vừa về vào buổi tối như thường ngày, mình chào đón anh thật to.
 
 Như chị chim lam, Al lảo đảo khi anh về trễ.
 
 "Al, anh mệt hả ?"
 
 "Ờ, chắc chắn. Nhưng thứ này sẽ hết sớm thôi. Về phòng nào."
 
 Al nắm tay mình và kéo thật nhanh. Mình chẳng có thời gian để nói nữa, và cố chạy để đừng bị bỏ lại.
 
 "Ồ, tao đi nhanh quá hả ?"
 
 "Không !", mình trả lời. Nhưng Al đi chậm lại một tí cho mình.
 
 % In the room, Al took off his jacket and sat on the bed as always. I climbed up next to Al and sat.
 
 Trong phòng, Al cởi áo khoát và ngồi lên giường như thường lệ. Mình trèo lên ngồi kế bên Al.
 
 "Hôm nay tao đi nói chuyện với một người quen. Nghe khá nhiều về thị trấn này, và cả chim lam nữa. Tao còn biết được nhiều thứ. Ta sẽ kết thúc vào ngày mai."
 
 "Al, anh có bạn nào không ?"
 
 Al nhăn mặt nhẹ.
 
 "... Chẳng ai cả"
 
 "Hừm... " Mình chẳng biết nói sao cả, nên mình đáp vậy. Có lẽ Al cũng là "người đáng thương".
 
 "... Vậy nghĩa là sao ? Sao cũng được ... Bây giờ, chuyện là thế này. Đầu tiên, Mischa, mày biết rồi đó, bả bị quên sạch mỗi ba ngày."
 
 "Đúng, nhưng chị ấy nhớ những thứ quan trọng."
 
 "Phải. Và có một gã thương tiếc cho căn bệnh của Mischa và muốn giúp."
 
 "Ngài quản gia ?"
 
 "Đúng rồi. Một ngày nào đó, Phil có năng lực để giúp Mischa. Và có một vài ảnh hưởng lớn cùng thiệt hại đâu đó."
 
 "Thiệt hại ?"
 
 "Chỉ là ... những thứ tệ hại, nói chung"
 
 Nghĩa là, ngài quản gia muốn giúp căn bệnh của Mischa, nhưng nó để lại hậu quả lớn cho thị trấn này ?
 
 "Hả, vì sao vậy ?"
 
 "Thứ Phil có là năng lực của rồng. Ý tao là, năng lực của rồng ngoài tầm hiểu biết của con người. Không gì ngạc nhiên nếu nó ảnh hưởng cả thị trấn. Và rồi, năng lực này bắt đầu ảnh hưởng thị trấn trùng hợp với Phil - chim lam, mày gặp rồi đó. Vậy là rất có khả năng chim lam là rồng."
 
 "Nhưng khi em hỏi chị chim lam "Chị có phải là rồng không ?", chị ấy nói "Chị là chim lam." Và không có quái vật nói dối xuất hiện."
 
 Al méo mặt khi tôi nhắc vậy.
 
 "Phải, đúng vậy... Đó là thứ mà tao sẽ hỏi chim lam trực tiếp. Quái, thế này thì ta chẳng tìm được gì đến khi tao nói chuyện với chim lam."
 
 "À, chim lam vừa trong vườn hôm này, và hôm qua, nên em cảm giác chị chim lam sẽ quay lại ngày mai. Có lẽ anh nói chuyện được."
 
 "À,... vậy là không chỉ một lần mỗi tuần. Được rồi, ngày mai tao sẽ nói chuyện với chim lam. Để xem nó có phải là rồng hay không và có phải Phil là bố của nó hay không ?"
 
 Nếu Phil là bố của chim lam ... ?
 
 "Hôm nay, chị chim lam nói Phil là người bạn quý."
 
 "Và không có quái vật nói dối ?"
 
 "Không, không hề !"
 
 "À, rồi ..."
 
 Al khoanh tay im lặng. Trông như anh đang lo lắng. Một hồi, anh thở dài và lăn lên giường.
 
 "Ta dừng ở đây và đi ngủ. Chúc ngủ ngon."
 
 Al nhắm mắt lại, nhưng mình chọi bụng anh.
 
 "Đợi đã, còn nữa, còn cái này nữa !"
 
 Còn một thứ mà mình vẫn tò mò. Al nhìn không hài lòng, nhưng vẫn ngồi dậy nhìn mình.
 
 "Gì nữa ?"
 
 "ừm, về năng lực của chị chim lam ? Nó giống em không ? Chị ấy có ăn được nói dối không ?"
 
 "Không, chim lam ... à, tao mệt rồi. Ngày mai."
 
 Và Al vùi đầu vào gối.
 
 Đó là thứ mình thực sự tò mò, nên mình lay Al, nhéo má, nhưng anh búng vào trán mình, nên mình ngoan ngoãn quay về giường.
 
 Ngày mai, bọn mình sẽ kết thúc vụ này. Sau đó, bọn mình sẽ đi khỏi thị trấn này.
 
 Mọi chuyện luôn thế này. Lần trước, và lần trước nữa. Một khi phi vụ kết thúc, bọn mình ra khỏi thị trấn, và hết. 
 
 Kết thúc là thứ đáng buồn. Ngực mình quặng một chút, nhưng mình vẫn nhắm mắt và đi ngủ. \\
 
 