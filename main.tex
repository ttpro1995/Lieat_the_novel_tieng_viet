\documentclass[13pt]{extarticle}

% language 
\usepackage[T5]{fontenc}
\usepackage[utf8]{inputenc}
\usepackage[vietnam,english]{babel}


\usepackage{indentfirst} % indent the first paragraph in section

\usepackage{float} % force picture to place [H] HERE

% allow enumitem with bracket
\usepackage{enumitem}% http://ctan.org/pkg/enumitem 

% make paragraph as subsubsubsection
\usepackage{titlesec}
\setcounter{secnumdepth}{4}
\titleformat{\paragraph}
{\normalfont\normalsize\itshape}{\theparagraph}{1em}{}
\titlespacing*{\paragraph}
{10pt}{3.25ex plus 1ex minus .2ex}{1.5ex plus .2ex}

\usepackage{graphicx} %load package for graphic

\usepackage{url} % allow url in bibtex

\usepackage{subcaption} % two figure side by side

\usepackage{pdfpages} % add the cover

% roman for section
\renewcommand\thesection{\Roman{section}}

% pdf metadata
\usepackage[pdftex,
pdfauthor={HahaTTpro},
pdftitle={Lieat},
pdfkeywords={Lieat},hidelinks]{hyperref}



% remove space between bulletpoint
\usepackage{enumitem}
\setlist{nolistsep,leftmargin=*}

\title{\Huge \textbf{Lieat} \\ Và cái gì nữa ở đây nè }
% Author
\author{\textsc{Tác giả Miwashiba} \\ Dịch sang tiếng Việt bởi HahaTTpro \\ Dịch từ bản tiếng Anh của Vgperson \thanks{\url{http://vgperson.tumblr.com/post/130927788020/lieat-the-novel-prologues-part-i}}}
% \date{Feb 31, 1990}

\begin{document}
	
	% tạo tiêu đề
	\maketitle
	
	% bỏ khoảng trống còn lại của trang
	\pagebreak
		
	% mục lục (table of content - toc)
	\tableofcontents
	
	\pagebreak
	
	\section{Con rồng ăn "nối dối" và kẻ lừa đảo}
	Nó không phải là một lời nói dối. Những gì đang xảy ra ngay bây giờ ... không phải bịa đặt.\\
	
	Tôi cố gắn giữ bình tĩnh, mở mắt thêm một lần nữa.\\
	
	Tôi thấy 1 bé gái. Với mái tóc vàng mượt mà, và đôi mắt là một sự pha trộn kỳ lạ của màu xanh và màu hồng. Từ nãy giờ, nó đang trố mắt nhìn xung quanh. Ừm, tôi đoán đó là phản xạ tự nhiên khi xuất hiện một cách đột ngột ở một nơi xa lạ.\\
	
	"Owie !"\\
	
	Tôi ném cái khăn tắm cho nó - chắc là ném hơi mạnh - nó kêu lên như thể bị cái khăn hất bay. Tôi không thể nhìn khi nó còn "trần truồng" như thế. Từ trong cái khăn tắm, nó ló đầu ra nhìn chằm chằm vào tôi. Tôi nhắm mắt lại và suy nghĩ. \\
	
	Tên tôi là Teobaldo Leonhearts. Tôi là "con người". \\
	
	Tôi làm công việc mà mọi người gọi tôi là "kẻ lừa đảo". Thật ra, tôi chẳng bao giờ dùng từ đó. \\
	
	Tại sao tôi lại giới thiệu tôi là "con người". Bởi vì, ngoài "con người", trong thế giới này, còn tồn tại một sinh vật kì lạ gọi là "rồng". \\
	
	Ngày xửa ngày xưa, xưa ơi là xưa, thế giới này có rất nhiều loài (giống người): người-thú, tiên, đủ thứ cả ... rồi họ thống nhất lại, và đều được gọi là "con người". Nhưng bây giờ, "rồng" được xem là một loài hoàn toàn tách biệt, và tất nhiên là có lý do. \\
	
	Mỗi con rồng đều có sức mạnh huyền bí của riêng nó. Nhiều người tin rằng, sức mạnh đó có thể thực hiện ước nguyện của con người, hoặc phát triển thành vũ khí bởi các quốc gia. \\
	
	Có nhiều bí ẩn vì sao những con rồng được sinh ra, cấu tạo sinh học của bọn chúng, nên thông tin liên quan đến rồng rất có giá trị. Tôi kiếm sống nhờ vào việc thu thập và bán những thông tin ấy cho bọn buôn tin. Đôi khi, tôi thấy việc này thú vị. \\
	
	Trở về vài ngày trước. \\
	
	Một cái gì đó màu xanh, tròn tròn như cái trứng xuất hiện dưới giường tôi. Tôi không biết thứ đó từ đâu ra, và tại sao nó nằm dưới giường của tôi. Bọn buôn tin nói có khả năng đó là trứng rồng. \\ 
	
	Sau khi tôi biết tin này, tôi cực kì sung sướng. Thử tưởng tượng coi bao nhiêu tiền có được từ việc bán thông tin này. Nó là một cái trứng vàng, đúng nghĩa. Tôi không ra ngoài một thời gian, để chắc chắn không ai biết gì về cái trứng này, để chăm sóc cho cái trứng và chờ nó nở... \\
	
	Tôi có nằm mơ cũng chẳng thể nào cái trứng nở nhanh như vậy. Nó nở ra một bé gái. Ý tôi là, tôi biết rồng có hình dạng giống người. Nhưng mà, tận mắt thấy cảnh một người nở ra từ quả trứng nó kinh dị không thể tả. \\
	
	Quan trọng hơn, tôi vẫn còn thấy phấn khích vì cơ hội ngàn vàn này và tôi quên để ý đến việc thu thập thông tin tốt về "thời khắc trứng rồng nở". Đó là điều tôi thấy hối hạn nhất đến giờ. \\
	
	"Bố ơi !!! ", con bé gọi
	
	"Đéo." 
	
	Tôi phủ nhận ngay lập tức. Đó không phải trứng tôi đẻ nhé. Nó tự xuất hiện. Nên, gọi tôi như vậy chẳng có nghĩa gì. Mà tại sao lại là "bố" ? \\
	
	"Bố !"\\
	
	Nó có nghe không đấy ? Hay là nó chẳng hiểu gì cả ? Nhưng mà, nghĩ lại thì, nó đáp bằng từ ngữ tôi hiểu được, ngay cả lúc mà tôi ném cho nó cái khăn tắm hồi nãy. \\
	
	TÔi nghe đồn rằng rồng khác người ở chỗ cách cơ thể và trí não của phát triển so với tuổi thật. Để có thể hiểu ngôn ngữ ngay sau nở, và đủ thông minh để nói chuyện ... tôi biết ơn điều này. \\
	
	"Này rồng, mày có năng lực gì ?" \\
	
	"Ai ? Ai đó ? Tôi ? Tôi là "mày"  ?", nó vừa đáp vừa nghiên cái đầu.\\
	
	À, ra là vậy. Tất nhiên nó chưa được đặt tên. \\
	
	"Ờ ... vậy thì ... Efina. Mày là Efina, hoặc Efi, cho ngắn" \\
	
	Đó là cái tên nhân vật trong quyển sách mà tôi vừa đọc. Thật rắc rối để nghĩ ra cái tên một cách nghiêm túc, mà cũng tốn thời gian nữa. Ý tôi là, nếu nó không thích thì, nó đổi tên khác. Nên đặt đại tên cho nó cũng chẳng có vấn đề gì. \\
	
	
	"Efi ... Efina ... Mình là Efina !" \\
	
	Nó vui vẻ lặp đi lặp lại cái tên và còn nhúng nhúng nữa. Lúc đó, cái khăn tắm bay đi, và tôi ngoảnh mặt đi chỗ khác. \\
	
	Thế là cuộc sống kỳ lạ của tôi với một con rồng đã bắt đầu.\\

	
	\section{Prologue}
		Bốn bàn chân bé tí ti 
	vểnh một tai như đang lắng nghe meo meo.
	A con mèo nó rất khôn 
	nó vểnh râu ngồi nghe em hát.
	A con mèo nó rất ngoan 
	suốt ngày chơi xung quanh cái vòng tròn. 
	
	\section{The Lie-Eating Dragon and the Forgotten-Color Songstress}
	
	
	\subsection*{Day 1}
	\addcontentsline{toc}{subsection}{Day 1} % add unnumber subsection to toc
	Ngày 1 
	
	\subsection*{Day 2}
	\addcontentsline{toc}{subsection}{Day 2} % add unnumber subsection to toc
	Ngày 2 
	
	\section{Efina’s Investigation}
	
	\section{Certain Dragons’ Memorandums}
	
	
\end{document}