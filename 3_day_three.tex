% Source http://vgperson.tumblr.com/post/130927788020/lieat-the-novel-prologues-part-i
%   \section{The Lie-Eating Dragon and the Forgotten-Color Songstress}
%	\subsection*{Ngày 3}

Vẫn buồn ngủ ... Sáng rồi hả ? Mình mở mắt và thấy ánh sáng chóa lòa. 

Sáng rồi. Mình ngồi dậy. Al dậy chưa nhỉ ?

Mình nhìn qua giường của Al. Anh vẫn đang ngủ.

"Sáng rồi, sáng rồi !". Mình gọi thật to. Đôi khi, mình nhảy lên giường Al để gọi anh dậy, nhưng mà anh sẽ la mình, nên mình không làm vậy một thời gian.

Al nhìn mình và nói một cách đâu khổ.

"... Mấy giờ rồi ?"

Mấy giờ ... ? Anh muốn mình xem đồng hồ hả ? Anh đã từng dạy mình cách xem đồng hồ, nhưng nó rối quá, mình chẳng nhớ.

"Xin lỗi, em chưa biết xem đồng hồ ..."

Al ngồi dậy một cách chậm chạp, dụi mắt. Tóc anh rối bù.

Al thay quần áo một cách loạng choạng, và vò tóc. Tất nhiên, ít nhất mình cũng có thể tự thay đồ.

"...Cái nơ bị lệch kìa"

... Mình \textit{gần như} có thể tự thay đồ. Mình được Al giúp một tẹo. Nên giờ mình đã sẵn sàng để đón ngày mới

Al lại bảo mình ở nhà, và ra khỏi phòng. Al chẳng cho mình theo hôm qua, và hôm trước cũng vậy. Nhất là khi mình rất muốn theo !

Mình phồng má và bĩu môi. Ở thị trấn nhỏ, Al dắt mình theo, nhưng khi ở thị trấn to như thế này, anh luôn bảo mình ở nhà.

Anh nói để mình khỏi lạc, nhưng mình chẳng bao giờ lạc ! ... Chắc vậy ...

Chỉ ngồi ở trong phòng một mình thì chán chết. Mình tự hỏi chị ca sĩ dậy chưa ? Chắc mình đi tìm chị ấy nói chuyện.

Sau vài cú nhảy thì cửa phòng cũng mở ra. Cái nắm cửa khá là cao đối với mình ...

Mình đẩy của và ra hành lang. Hôm qua, mình bị lạc ở đây, và thế là một cuộc phiêu lưu bắt đầu. Biệt thự này to thật. Nhưng lần này, chắc mình sẽ ổn.

Khi chị ca sĩ thức dậy, chị sẽ xuống cái quảng trường rộng lớn tầng dưới. Hôm qua, ngài quản gia Phil bảo đó là cái "phòng tiệc "

Có một cái bàn tròn thật to với ghế xếp xung quanh, và chị ấy sẽ ngồi ở đó nói chuyện với cư dân, hoặc hát.

Chị ấy trông rất vui. Và mọi người cũng rất vui.\\

% I went downstairs to the banquet hall, but she wasn’t there yet. But mister butler was there cleaning instead.

Mình xuống phòng tiệc tầng dưới, nhưng chị ấy chưa đến. ngài quản gia đã đến, và ngài đang lau dọn.

"À, chào buổi sáng Efi". Ngài quản gia chào mình với nụ cười.

"Chào buổi sáng", mình chào lại thật to. Mình không để thua đâu. Tiếng của mình vọng khắp biệt thự như được hét từ trên núi.

"Chị ca sĩ đâu rồi ?"

"À, anh tin rằng Mischa đang tắm. Hôm nay chị ấy trông rất vui."

Vậy là chị ca sĩ đang tắm. Tắm sáng giúp tâm trạng sảng khoái ? Mình chỉ tắm vào buổi tối thôi, mình không biết.

"Vậy chúng ta cùng nói chuyện nhé, anh Phil !" Nếu chị Mischa chưa tới, thì nói chuyện với ngài quản gia cũng được. Anh nhìn vào cây chổi vài giây, rồi thở dài.

"Cũng được, anh sắp xong rồi."

Tuyệt ! Đây là lần đầu tiên mình thực sự nói chuyện với ngài quản gia. Anh ấy thường xuyên đi đây đó trong biệt thự. Và mình sẽ biết nhiều thứ từ anh hơn chị ca sĩ.

"Chờ tí", anh bảo, nên mình ngồi vào ghế xung quanh cái bàn tròn và chờ. Một chốc sau, ngài quản gia quay lại với cái đĩa và cái tách.

"ồ, đồ ngọt !"

Trong đĩa là loại đồ ngọt mình chưa từng thấy bao giờ. Nó nhiều màu, mập mạp và còn rất dễ thương nữa.

Thấy mắt mình lấp lánh, anh nói. "Em chưa thấy món này bao giờ hả ? Đây gọi là \textit{macarons}. Tiệm bánh kẹo gần đây bán đó. Ngon lắm !"

Anh bỏ một cục vào tay mình. "Nè"

Mình lăn nó. Cục macaron lăn tròn và nhảy nhót trong tay mình.

Rồi, mình cắn một miếng. Nó mềm mịn và ngọt lịm trong miệng mình. Ngon lắm !

"Em có vẻ thích. Thật tuyệt". Ngày quản gia nói, và đưa mình cái tách. Cái tách đầy trà. Mình biết, vì mình đã uống trà rồi.

"Em muốn bao nhiêu đường"

"Nhiều"

Câu trả lời của mình làm ngài quản gia ngạc nhiên, và mỉn cười một cách trẻ con. Anh bỏ hai viên đường vào tách trà và quậy lên bằng cái muỗng nhỏ. Và anh ngồi vào ghế kế bên mình.

"Này anh quản gia, có phải anh và chị ca sĩ là bạn thân lâu năm ?". Mình hỏi, đột nhiên, mình thấy lo. Anh nói anh là quản gia của chị ca sĩ, nhưng mình thấy họ khá thân nhau.

"Đúng vậy. Anh và chị ấy là bạn đã khá lâu trong thị trấn này." Anh trả lời và rót trà.

"Hồi trước chị ấy có thích hát không ?"

"Có. Chị ấy hát thường xuyên. Anh rất thích nghe giọng hát đó."

Anh nở nụ cười ấm áp, và mặt anh rạng rỡ ánh bình minh. Mình thấy anh quản gia rất thích chị ca sĩ này.

Mình còn chắc rằng anh không nói dối, và mình thấy ấm trong lòng.

Khi người ta nói sự thật, nhất là nói về thứ họ thích, mình sẽ thấy vui. Mình thích cảm giác này.

Mặc dù, nếu chẳng ai nói dối, mình sẽ chết đói mất.

"Em nghe Al nói là khi ta quen ai đó lâu, thỉnh thoảng ta sẽ có xích mích. Anh đã bao giờ cãi nhau với chị ca sĩ chưa ?"

"ờ ..."

Ngài quản gia cầm cái tách lên định uống, thì bỗng đặt cái tách xuống khi nghe mình hỏi.

"Nghĩ lại thì bọn anh ... chưa cãi nhau lần nào. Nói lí lẽ thì, trường hợp này cũng có thể xảy ra."

Có lí hả ? Anh ấy có ý gì ? Mình thắc mắc cách mà ngài quản gia trả lời. 

Mình không cảm thấy anh nói dối. Mình không chắc nữa. Không có Al ở đây thì mình không thể chắc chắn.

"Chào buổi sáng"

Khi mình dăm chiêu suy nghĩ thì chị ca sĩ bước xuống với làn sương mờ ảo xung quanh.

"Chào chị"

Chị chớp mắt và nhìn mình.

"Chào buổi sáng, ... ờ, Efi". Mất một hồi để chị kêu tên của mình rồi mỉn cười.

Chỉ là ngập ngừng một tẹo. Nhưng mình thấy cái ngập ngừng này có gì đó không ổn. \\

% Continuing on from yesterday, I went around town for several hours, but failed to obtain any useful information.

