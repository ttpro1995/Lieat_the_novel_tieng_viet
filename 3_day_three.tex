% Source http://vgperson.tumblr.com/post/130927788020/lieat-the-novel-prologues-part-i
%   \section{The Lie-Eating Dragon and the Forgotten-Color Songstress}
%	\subsection*{Ngày 3}

Vẫn buồn ngủ ... Sáng rồi hả ? Mình mở mắt và thấy ánh sáng chóa lòa. 

Sáng rồi. Mình ngồi dậy. Al dậy chưa nhỉ ?

Mình nhìn qua giường của Al. Anh vẫn đang ngủ.

"Sáng rồi, sáng rồi !". Mình gọi thật to. Đôi khi, mình nhảy lên giường Al để gọi anh dậy, nhưng mà anh sẽ la mình, nên mình không làm vậy một thời gian.

Al nhìn mình và nói một cách đâu khổ.

"... Mấy giờ rồi ?"

Mấy giờ ... ? Anh muốn mình xem đồng hồ hả ? Anh đã từng dạy mình cách xem đồng hồ, nhưng nó rối quá, mình chẳng nhớ.

"Xin lỗi, em chưa biết xem đồng hồ ..."

Al ngồi dậy một cách chậm chạp, dụi mắt. Tóc anh rối bù.

Al thay quần áo một cách loạng choạng, và vò tóc. Tất nhiên, ít nhất mình cũng có thể tự thay đồ.

"...Cái nơ bị lệch kìa"

... Mình \textit{gần như} có thể tự thay đồ. Mình được Al giúp một tẹo. Nên giờ mình đã sẵn sàng để đón ngày mới

Al lại bảo mình ở nhà, và ra khỏi phòng. Al chẳng cho mình theo hôm qua, và hôm trước cũng vậy. Nhất là khi mình rất muốn theo !

Mình phồng má và bĩu môi. Ở thị trấn nhỏ, Al dắt mình theo, nhưng khi ở thị trấn to như thế này, anh luôn bảo mình ở nhà.

Anh nói để mình khỏi lạc, nhưng mình chẳng bao giờ lạc ! ... Chắc vậy ...

Chỉ ngồi ở trong phòng một mình thì chán chết. Mình tự hỏi chị ca sĩ dậy chưa ? Chắc mình đi tìm chị ấy nói chuyện.

Sau vài cú nhảy thì cửa phòng cũng mở ra. Cái nắm cửa khá là cao đối với mình ...

Mình đẩy của và ra hành lang. Hôm qua, mình bị lạc ở đây, và thế là một cuộc phiêu lưu bắt đầu. Biệt thự này to thật. Nhưng lần này, chắc mình sẽ ổn.

Khi chị ca sĩ thức dậy, chị sẽ xuống cái quảng trường rộng lớn tầng dưới. Hôm qua, ngài quản gia Phil bảo đó là cái "phòng tiệc "

Có một cái bàn tròn thật to với ghế xếp xung quanh, và chị ấy sẽ ngồi ở đó nói chuyện với cư dân, hoặc hát.

Chị ấy trông rất vui. Và mọi người cũng rất vui.\\

% I went downstairs to the banquet hall, but she wasn’t there yet. But mister butler was there cleaning instead.

Mình xuống phòng tiệc tầng dưới, nhưng chị ấy chưa đến. ngài quản gia đã đến, và ngài đang lau dọn.

"À, chào buổi sáng Efi". Ngài quản gia chào mình với nụ cười.

"Chào buổi sáng", mình chào lại thật to. Mình không để thua đâu. Tiếng của mình vọng khắp biệt thự như được hét từ trên núi.

"Chị ca sĩ đâu rồi ?"

"À, anh tin rằng Mischa đang tắm. Hôm nay chị ấy trông rất vui."

Vậy là chị ca sĩ đang tắm. Tắm sáng giúp tâm trạng sảng khoái ? Mình chỉ tắm vào buổi tối thôi, mình không biết.

"Vậy chúng ta cùng nói chuyện nhé, anh Phil !" Nếu chị Mischa chưa tới, thì nói chuyện với ngài quản gia cũng được. Anh nhìn vào cây chổi vài giây, rồi thở dài.

"Cũng được, anh sắp xong rồi."

Tuyệt ! Đây là lần đầu tiên mình thực sự nói chuyện với ngài quản gia. Anh ấy thường xuyên đi đây đó trong biệt thự. Và mình sẽ biết nhiều thứ từ anh hơn chị ca sĩ.

"Chờ tí", anh bảo, nên mình ngồi vào ghế xung quanh cái bàn tròn và chờ. Một chốc sau, ngài quản gia quay lại với cái đĩa và cái tách.

"ồ, đồ ngọt !"

Trong đĩa là loại đồ ngọt mình chưa từng thấy bao giờ. Nó nhiều màu, mập mạp và còn rất dễ thương nữa.

Thấy mắt mình lấp lánh, anh nói. "Em chưa thấy món này bao giờ hả ? Đây gọi là \textit{macarons}. Tiệm bánh kẹo gần đây bán đó. Ngon lắm !"

Anh bỏ một cục vào tay mình. "Nè"

Mình lăn nó. Cục macaron lăn tròn và nhảy nhót trong tay mình.

Rồi, mình cắn một miếng. Nó mềm mịn và ngọt lịm trong miệng mình. Ngon lắm !

"Em có vẻ thích. Thật tuyệt". Ngày quản gia nói, và đưa mình cái tách. Cái tách đầy trà. Mình biết, vì mình đã uống trà rồi.

"Em muốn bao nhiêu đường"

"Nhiều"

Câu trả lời của mình làm ngài quản gia ngạc nhiên, và mỉn cười một cách trẻ con. Anh bỏ hai viên đường vào tách trà và quậy lên bằng cái muỗng nhỏ. Và anh ngồi vào ghế kế bên mình.

"Này anh quản gia, có phải anh và chị ca sĩ là bạn thân lâu năm ?". Mình hỏi, đột nhiên, mình thấy lo. Anh nói anh là quản gia của chị ca sĩ, nhưng mình thấy họ khá thân nhau.

"Đúng vậy. Anh và chị ấy là bạn đã khá lâu trong thị trấn này." Anh trả lời và rót trà.

"Hồi trước chị ấy có thích hát không ?"

"Có. Chị ấy hát thường xuyên. Anh rất thích nghe giọng hát đó."

Anh nở nụ cười ấm áp, và mặt anh rạng rỡ ánh bình minh. Mình thấy anh quản gia rất thích chị ca sĩ này.

Mình còn chắc rằng anh không nói dối, và mình thấy ấm trong lòng.

Khi người ta nói sự thật, nhất là nói về thứ họ thích, mình sẽ thấy vui. Mình thích cảm giác này.

Mặc dù, nếu chẳng ai nói dối, mình sẽ chết đói mất.

"Em nghe Al nói là khi ta quen ai đó lâu, thỉnh thoảng ta sẽ có xích mích. Anh đã bao giờ cãi nhau với chị ca sĩ chưa ?"

"ờ ..."

Ngài quản gia cầm cái tách lên định uống, thì bỗng đặt cái tách xuống khi nghe mình hỏi.

"Nghĩ lại thì bọn anh ... chưa cãi nhau lần nào. Nói lí lẽ thì, trường hợp này cũng có thể xảy ra."

Có lí hả ? Anh ấy có ý gì ? Mình thắc mắc cách mà ngài quản gia trả lời. 

Mình không cảm thấy anh nói dối. Mình không chắc nữa. Không có Al ở đây thì mình không thể chắc chắn.

"Chào buổi sáng"

Khi mình dăm chiêu suy nghĩ thì chị ca sĩ bước xuống với làn sương mờ ảo xung quanh.

"Chào chị"

Chị chớp mắt và nhìn mình.

"Chào buổi sáng, ... ờ, Efi". Mất một hồi để chị kêu tên của mình rồi mỉn cười.

Chỉ là ngập ngừng một tẹo. Nhưng mình thấy cái ngập ngừng này có gì đó không ổn. \\

\begin{center}
	******
\end{center}

% Continuing on from yesterday, I went around town for several hours, but failed to obtain any useful information.

Tiếp hôm qua, tôi đã đi quanh thị trấn này vài giờ, và chẳng tìm được tin gì có ích.

"Chim lam có tồn tại không nhỉ ?"

Tôi cũng bắt đầu nghi ngờ. Mấy người dân tôi hỏi đều trả lời như nhau.

"Nhờ chim lam, chúng tôi hạnh phúc". Dù cách diễn đạt có khác nhau, họ đều trả lời thế, với vẻ mặt vui vẻ.

Chắc chắn, chim lam là truyền thuyết phổ biến trong thị trấn này. Nhưng có gì đó không ổn ?

Chắc chim lam là thể loại thần tượng gì đó. Tôi không hiểu nổi có kiểu tôn giáo từ thứ này. Người ta đổ thừa điều tồi tệ đến với họ là do trời, hoặc dấng tối cao nào đó. Và ngược lại thì cũng chẳng lạ gì.

"... Dù sao thì, mình cũng mất phương hướng rồi."

Còn một lựa chọn khác. Hôm qua, tôi nghe lời ông già có râu, và né...

Những cư dân bất hạnh ...

Từ hôm qua tới giờ, tôi toàn chọn trò chuyện với cư dân hạnh phúc. Bởi vì, cái câu "họ là yêu quái, hãy coi chừng" làm tôi chùng bước. Tôi luôn muốn tránh rắc rối.

Nhưng giờ thì tôi bí rồi. Và tôi cảm giác là nếu tôi cứ tránh họ như vậy thì chuyến đi tới thị trấn này chỉ tổ tốn thời gian.

Vậy không còn cách nào khác. Tất nhiên, ít ra thì tôi đủ mạnh để tự vệ.

Tôi kiểm tra ám khí cái túi áo khoác và túi quần. Tôi đi vào con hẻm chật hẹp để tìm những cư dân bị áp bức.\\

Thở dài...

Tôi ngồi dưới bóng cây đại thụ.

Tôi đã đi vòng vòng thị trấn khá nhiều. Đã lùng sục mọi ngỏ ngách, từ hẻm nhỏ đến đường lớn. Và mấy cư dân buồn bả tôi đã từng thấy giờ chẳng thấy đâu, một người cũng không.

Chắc họ cũng né tôi luôn. Tôi ráng nhớ lại mạnh mối. Có mấy mụ già buôn dưa lê trong ngày đầu tiên.

"Cho chúng ở trong biệt thự ấy có ổn không ?" - Tôi chắc mấy mụ nói vậy.

Người sống trong biệt thự, Mischa và Phil... Tôi chưa thấy có gì lạ với hai người đó. Và những người đến và đi từ căn biệt thự... dù họ tỏ vẻ tôn sùng giọng hát của Mischa, họ đều vô tư, vui vẻ, không gì đáng ngờ.

... "Tất cả bọn người vui vẻ" làm tôi lo. Có thể, những người buồn bã chưa vào giờ nghĩ về việc thăm nhà ai đó.

Khi tôi đang suy nghĩ, một người đi ngang. Tôi nấp sau cây và quan sát. Một chàng trai trẻ với cái mặt buồn như chó cắn.

"... Cho tôi xin một phút."

Tôi nhanh chân chặn đường cậu ta và mỉn cười. Cậu ta ngạc nhiên, và khẽ hét lên

"uhhh"

"Tôi đang đi du lịch. Tôi chỉ muốn hỏi vài điều. Tôi sẽ hậu tạ.". Tôi nói một cách yếu đuối. Tôi cố diễn mình là một tên yếu đuối. Việc này khiến người ta tin bạn, miễn là họ đừng thấy mình có âm mưu gì hết.

"Không... tôi không cần hậu tạ gì cả. Nếu chỉ vài giây, thì được". Chàng trai trẻ trả lời.

Cậu ta không muốn nói chuyện ở đây, nên dắt tôi tới cái ghế dài gần đó. Tôi đi theo mà chẳng nói năng gì. \\

% Once we sat down on the bench, the young man started talking before I could

Khi chúng tôi yên vị, chàng trai trẻ bắt đàu nói trước khi tôi kịp mở miệng.

"Ừm, anh là khách du lịch đến đây từ 2 hôm trước hả, đúng không ? Anh trông như họ nói. Có thật là chúng để anh ở lại căn biệt thự ?"

Tôi như muốn ói. Tôi không ngờ cậu ta bắt đầu nói, và nói về cái đó.

"Ờ, đúng vậy ... có gì đó với căn biệt thự à ?"

"Vâng, đúng là căn biệt thự đó, nhưng gần như có gì đó không ổn với cả thị trấn ...". Cậu ta cuối đầu với vẻ mặt âm u.

"Có gì không ổn ?"

Tôi để ý ngay, nhưng hỏi trực tiếp đôi khi phản tác dụng. Nên tôi chỉ lập lại và đợi nghe thêm từ miệng cậu ta.

"... Có những cư dân hạnh phúc, anh biết đấy. Họ vô tư một cách đáng sợ. Có tin đồn giữa bọn tôi rằng, họ có thể là yêu quái."

Tôi choáng váng. Vậy là bọn người buồn bã nói cùng một thứ về bọn người vui vẻ. Mỗi bên đều nghi ngờ bên kia là yêu quái, hay quái vật các kiểu.

"Gần đây, chúng tôi nghe rất nhiều chuyện là. Ông nội của ai đó đã chết đi ngang như không có gì, một bạn gái đã chuyển đi xa đột nhiên xuất hiện trên ghế và cười... Nên mọi người mệt mỏi. Có cả tin đồn nữ ca sĩ trong căn biệt thự đang gọi hồn với giọng hát của cô ta."

"Ai đó chuyển đi chưa hẳn đã chết. Nó chẳng lạ gì nếu người ta quay lại vì công việc, phải không ?"

"ờ ... Thực ra, đó là bồ cũ của tôi. Chúng tôi chia tay và cô ấy chuyển đi thật xa, nên tôi không biết cô ấy còn sống hay chết. Nhưng tôi chắc rằng cô ta chẳng có việc gì ở đây."

Mặc dù vậy, tôi chẳng có lý do nào để tin chuyện ma quỷ gì ở đây. Suy cho cùng thì theo quan điểm của những người buồn bã, chẳng ngạc nhiên khi họ tin vậy.

Những thứ đáng ngờ đang diễn ra quanh đây. Vậy nên tôi đi tìm lời giải với ít nhất thì cũng ăn khớp với nhau, tôi muốn chứng minh rằng, mọi người chưa đánh mất chính mình.

"Vậy nên không một ai đến gần biệt thự đó, hoặc những người vui vẻ. Chúng tôi nghe nói họ để anh ở lại biệt thự, và ... ừm, hơi khó hỏi trực tiếp, nhưng ... Tôi tự hỏi anh là ma hay là ..."

À ha, đó là lý do tại sao những người buồn bã không dám đến gần tôi, và tôi chẳng tìm ra ai cả khi đi khắp nơi. Chắc chắn, đây là thị trấn lớn. Nhưng hai ngày cũng đủ để tin đồn lan truyền đến hầu hết mọi người. Và khi họ càng muốn tránh xa tôi thì những người chưa quen với thị trấn như tôi càng khó tìm ra họ.

"Ra là thế ... Nhưng không, tôi không phải ma. Muốn tự thử không ?"

"Ồ, tôi nhận ra khi nói chuyện với anh. Mặc dù, lúc đầu tôi hoảng sợ khi anh bắt chuyện tôi, lo rằng anh sẽ đem tôi qua thế giới bên kia..."

Chàng trai trẻ cười nhăn nhó. Tôi cũng vậy, tôi ngộ ra rằng cậu ta chắc hẳn sẽ tin bất kì điều gì một cách nhanh chóng. Nhưng tôi không thể trách cậu ta gặp khó khăn khi ra quyết định một cách điềm tĩnh vì cậu ta xanh xao quá.

"Tôi còn muốn hỏi thêm một thứ. Anh có biết gì về chim lam ?"

"Chim lam ... ờ, có. Có một truyền thuyết về nó trong thị trấn này từ lâu rồi. Nhưng chỉ là truyền thuyết. Ngay cả khi có tin đồn lan truyền rộng rãi rằng chim lam tồn tại thực sự ... Tin đồn đó bắt đầu khi tôi nghe tin đồn về mấy thứ kì lạ khác. Ngay cả khi chim lam có thật, nó chẳng mang lại hạnh phúc gì."

Mặt chàng trai trẻ ấy càng nói càng u sầu. Tôi không nghĩ cậu ta nói xạo. Nên, những gì cậu ta nói rất đáng tin.

Và tin đồn về chim lam bắt đầu cùng thời điểm, cũng đáng tin luôn.

"Xin cảm ơn rất nhiều vì đã kể cho tôi"

"Không thành vấn đề... Nhưng hãy bảo trọng, hỡi người lũ hành. Bởi vì anh có thể bị kéo sang thế giới bên kia lúc nào không hay đấy, không đùa đâu."

Rồi cậu ta đúng dậy, cuối chào vài cái, và bỏ đi thật nhanh.

Rồi, tôi đã nói chuyện với những người buồn bã luôn. Và có vẻ như cư dân thị trấn này nghi ngờ lẫn nhau là ma quỷ, yêu quái, quái vật các kiểu.

Tôi không ngờ sự việc lại thế này, nhưng tôi trở nên bị thuyết phục rằng chim lam có dính líu tới vụ này một cách nào đó.

"Có vẻ như thứ đang mang lại hạnh phúc không dễ thương tí nào, chắc luôn"

Tôi nhúng vai và dựa lên ghế dài. Tôi quan sát xung quanh, lấy ra cái hộp và cái bật lửa từ trong túi, và kiểm tra cái hộp. Còn vài điếu thuốc.

Tôi lấy một điếu ra và châm lửa. Tôi từng hút thuốc liên tục, nhưng bắt đầu cắt giảm khi Efi xuất hiện. Kẻ buôn tin đầy lông vũ bảo tôi khói thuốc không tốt cho rồng. 

Tôi tự hỏi khả năng khè lửa của Efi có thể hữu dụng trong tình huống thế này, và tôi sẽ nhờ nó châm thuốc. 

Nói cho đủ thì, nhiều hơn chỉ một điếu thuốc bị cháy.

Nghĩ đi nghĩ lại, chẳng hiểu sao tôi lại nghĩ ra thí nghiệm ngu ngốc này. Chỉ vì tò mò ... Thứ này dạy tôi rằng tò mò nguy hiểm như thế nào.

"... Không ngon lắm"

Nhìn làn khói trắng lượn lờ, tôi lờ đờ ngẫm nghĩ.

Có thể tại nhãn hiệu khác, hoặc tôi mua nhầm loại, nhưng tôi chẳng thấy tốt hơn tí nào vì tôi bỏ lâu rồi. Tôi quyết định không phí thời gian nữa.

Tôi thở dài, và dụi điếu thuốc đang hút dở vào khay đựng tàn thuốc bên ghế dài, và nhét gói thuốc vào túi. Mặt trời đã khá thấp.

Hôm nay thế là đủ. Cảm giác vẫn còn nhiều việc để làm. Nhưng tôi không muốn về tay không khi đã đi được đến nước này.

"Tao sẽ tìm ra mày, tao thề đó..."

Chỉ cần tôi thấy không nguy hiểm tới cái mạng này thì tôi sẽ theo đến cùng. Tôi sẵn sàng đặt cược mọi thứ, ngoại trừ cái mạng của mình. Tôi sẽ tìm mọi thông tin chi tiết về thứ này, và kiếm bộn tiền từ kẻ buôn tin đội mũ. Không có lí nào tôi đến đây và chẳng có gì để khoe.

Ý nghĩ đó xoa dịu trí óc tôi tốt hơn bất kì điêu thuốc nào \\

\begin{center}
	******
\end{center}
% "Oh, welcome back. You were out quite late..."
"Ồ, chào mừng trở về. Anh về khá muộn ..."

Khi tôi về tới căn biệt thự, Mischa, ngồi trên ghế, quay lại và chào đón tôi một cách ấm áp. Tôi chắc chắn Efi sẽ nhảy vồ vào tôi như hôm qua, nên tôi cũng khá ngao ngán.

"À, chào ... ủa ? Efi đâu rồi ?"

Tôi nhìn quanh sảnh và chẳng thấy dấu hiệu gì của nó ? Nó lạc rồi sao ?

"Ồ, không việc gì phải lo". Mischa nói từ phía bên kia của bàn tròn, ngoài tầm mắt tôi.

Tôi nhìn qua và thấy Efi đang ngủ trên đùi của Mischa.

"... Xin lỗi, có vẻ như chúng tôi làm phiền các bạn..."

"Ồ không, con bé chỉ mệt khi nói chuyện với tôi. Đó là lỗi của tôi."

Cô ấy cười khúc khích trả lời, nhưng nó chẳng có gì đáng cười với tôi ... Tôi nhìn nó đang ngủ ngon lành.

Trước khi tôi quyết định đánh thức Efi, Mischa hỏi một câu.

"Hôm nay hình như anh đi bộ khá nhiều. Anh đang tìm gì à ?"

"Vâng, đúng thế. Trông tôi mệt mỏi lắm à ... ?"

"Tại sao, đúng là vậy. Tôi thấy anh trông khá mệt so với chỉ đơn thuần là đi ngắm cảnh."

Tôi cố nở nụ cười, nhưng một ít mệt mỏi của tôi cũng hiện lên luôn. Không hẳn là tôi lớn lên trong tình cảnh này, nhưng tôi cảm giác là những ngày qua, tôi thực sự mệt mỏi.

"Tôi làm cô tò mò đến thế à ?", tôi hỏi và cố mỉn cười, và cô ấy gật đầu.

"Hahaha, xin lỗi. Cô đang có thời gian thư giản thoải mái thì tôi xuất hiện như thế này"

"Ồ, đừng bận tâm."

Mischa cười, nhưng tôi cảm thấy có gì đó về cô ta.

"Vâng, tôi sẽ đi ngủ sớm...cùng Efi."

Tôi lay nó dậy, và lắc luôn cả Mischa, vậy nên tôi nhẹ nhàng bồng Efi lên. Rồi tôi vác nó lên vai, nó vẫn chưa dậy. Nó hoàn toàn bất tỉnh.

"Sao anh vác em ấy kì vậy ?", Mischa hỏi khi thấy tôi vác Efi.

"Đừng lo, chuyện thường ngày."

Tôi cuối chào và leo lên cầu thang.\\


% "...Are you really awake?"







