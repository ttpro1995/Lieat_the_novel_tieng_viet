% Source http://vgperson.tumblr.com/post/130927788020/lieat-the-novel-prologues-part-i
%   \section{The Lie-Eating Dragon and the Forgotten-Color Songstress}
%	\subsection*{Ngày 2}

Tiếng chim hót như cù vào tai tôi và đánh thức tôi dậy.

Tôi sợ cái mệt mỏi hôm qua vẫn còn đeo bám tôi. Nhưng tôi ngồi dậy và nhận ra mình đã bình phục hoàn toàn.

Tôi tự hào nói rằng tôi chẳng có cuộc sống tử tế kể từ khi tôi được sinh ra, nhưng bù lại cho tôi khả năng bình phục. Cuộc sống cũng không quá bất công. Tuổi trẻ thật tuyệt vời. Tôi mở toan cửa sổ và tận hưởng làn gió nhẹ ùa vào. 

Khi tôi còn đang tận hưởng buổi sớm sảng khoái, Efi dậy với một cái ngáp. Và cảm xúc dâng trào về buổi sớm tinh mơ kết thúc tại đây.

"... Hôm nay mày dậy sớm nhỉ !"

Tôi nói câu đầu tiên với Efi cùng vẻ mặt đơ đơ, chẳng còn chút sảng khoái nào. Nó dụi mắt, và nói "Chào buổi sáng ...". Tôi đáp "Chào buổi sáng ..."

"Hôm nay chúng ta đi đâu ?", Efi hỏi với giọng ngáy ngủ.

"Xem nào, tao sẽ vào thị trấn điều tra. Còn mày ở yên đây, trong căn biệt thự này ..."

"Không!", nó từ chối khi tôi còn chưa nói xong. Cảm giác như nó phản đối kịch liệt với ý sau của câu vừa rồi. Chẳng lẽ nó đã đến tuổi nổi loạn ?

"Chúng ta sẽ đi cùng nhau !" Nó nói thêm. Khi tôi còn nhỏ, tôi chẳng bao giờ cứng đầu thế này. Nhưng, tôi không phải là bé gái.

Nói thật thì, nếu chúng tôi bị tách ra trong cái thị trấn rộng thế này, và nó bị lạc, nó sẽ làm gì ? Nó có hiểu nó là rồng, thứ gì đó mà mọi người luôn theo đuổi ?

"Tao sẽ mua gì đó cho mày"

"Vâng ạ"

... Không thể xác định được nó dễ hay khó bảo. Ồ, hay. Cuộc chiến cũng chấm dứt. Tôi thả lỏng vai.

Nghĩ lại thì, tôi nghe nói rồng không thỏa mãn với thức ăn của con người. Thế thì tại sao Efi thích đồ ngọt đến vậy ? Trước khi đến đây, tôi đã cho nó ăn khá nhiều "nói dối", nên tôi không nghĩ nó đói sớm đến vậy.

"Kẹo chắc sẽ ngon"

"Chắc chắn", tôi trả lời.

Có rất nhiều bí ẩn về bé rồng này. \\

\begin{center}
	******
\end{center}

% To be extra sure, I asked the butler Phil to keep an eye on Efi, and left her at the mansion while I explored the town.

Để chắc ăn, tôi nhờ quản gia Phil trông chừng Efi, và để nó trong căn biệt thự khi tôi đi khám phá thị trấn này.

Hôm nay, tôi mới thực sự bắt đầu tìm chim lam.

Ta nên bắt đầu từ đâu nhỉ ... tôi đang nghĩ ngợi. Bỗng ai đó đập vào lưng tôi.

"Này, nhà lữ hành. Thật tuyệt vời khi gặp lại cậu tại đây. Tôi đoán là, bởi vì hàng ngày tôi chỉ quanh quẩn những chỗ quen. Ga ha ha ha ha"

Tôi quay lại và thấy ông già có râu lúc đầu tôi gặp. Ở thị trấn rộng lớn và đông dân này, chúng tôi đã gặp nhau khá nhiều lần.

Qua những cuộc nói chuyện với lão, có vẻ như lão biết khá nhiều về thị trấn này. Có thể lão biết gì đó về chim lam.

"Ồ, chúc một ngày tốt lành, ta cứ tình cờ gặp nhau nhỉ. Ngài thấy đó, tôi đến đây tìm chim lam. Ngài có biết gì về nó không, thưa ngài ?" Tôi hỏi với giọng vui vẻ và mỉn cười, thậm chí tôi thấy mình có vẻ ngáo.

Thậm chí tôi còn có thể ngáo hơn thế nữa.

"À, chim lam. Xem nào. Vâng, nó ở thị trấn này, đúng rồi. Nhưng mà, tôi vẫn chưa bao giờ thấy nó. Tuy nhiên, nó là lý do tại sao mọi người ở thị trấn này lại hạnh phúc, với vài ngoại lệ !"

"Một vài ... ngoại lệ ?"

Tôi sững sốt. Nó có thể liên quan đến bọn lập dị mà tôi gặp khi tồi vừa đến đây.

"À, đừng nói lớn quá ..."

Ông lão nghiên người, để tay che miệng lại và đưa lại gần tai tôi.

"Cậu không thấy thị trấn này ... có gì đó không ổn ? Hầu hết mọi cư dân đều hạnh phúc và vô tư. Nhưng tôi chắc chắn là cậu đã thấy một vài người trông đáng ngờ."

Điều này khiến tôi nhớ lại cuộc đối thoại của mấy quý bà, à không, mụ già, hôm qua. Thể loại ấy khá hiếm trong thị trấn này, nhưng sự thật, nó tồn tại ở đây.

"Thấy chưa, có tin đồn rằng chúng là yêu quái đội lốp người. Và ngày nào đó, nó sẽ hiện nguyên hình, và công kích thị trấn xinh đẹp này. Nên nhưng người lữ hành như cậu, hãy cảnh giác bọn chúng."

Khi lão vừa nói xong, lão duỗi lưng, uốn mình và phá lên cười. Tôi bối rối với sự thay đổi tâm trạng nhanh đến chóng mặt này.

"Sao, thực ra, chẳng có gì đáng lo cả. À, cậu có nói rằng cậu muốn tìm chim lam ? Được rồi, tôi nghe nói chim lam sẽ xuất hiện một lần mỗi tuần."

"Một lần mỗi tuần ... ?"

Mặt dù lão nói "nó ở thị trấn này", có vẻ như bạn khó gặp nó thường xuyên. Vậy, chắc là sẽ khó xác thực tin này trong thời gian ngắn đây. Tôi vừa nghĩ ngợi, vừa cảm ơn lão và bỏ đi.

Chắc tôi phải ở lại thị trấn này cả tuần quá, vì tôi chẳng biết khi nào chim lam xuất hiện. Chắc tôi nên đem thêm một ít tiền nữa.

Từ khi tôi sống với Efi, thì ngân quỹ tôi luôn bị thâm hụt. Khi tôi còn ở một mình, tôi tự do quyết định khi nào bỏ bữa. Nhưng giờ thì không được. Tôi cảm giác mình trở thành bà nội trợ dã chiến đang vật lộn với vấn đề tài chính.

Đây là trải nghiệm mà tôi chưa bao giờ nghĩ mình sẽ trải qua trước khi gặp Efi.

"Nhức đầu quá ..."

Tôi dùng ngón tay xoa thái dương cho đến khi cơn nhức đầu thuyên giảm.

Nghĩ lại thì, gã đó bảo "thị trấn này có cư dân hạnh phúc và vô tư, và những kẻ lập dị thì không ...". Và còn tin đồn có lũ yêu quái đang ẩn mình nữa...\\

% After walking a bit, in a main street lined with various stores, I decided to try casually observing the townspeople.

Khi đi được một lát, tôi đến đoạn đường với đủ loại cửa hàng hai bên, tôi quyết định quan sát cư dân ở đây.

Đúng là, phần lớn cư dân mặt mày sáng sủa, nhưng lâu lâu thấy có vài người có nét mặt tâm tối đến đáng ngờ. Và tôi không chỉ thấy một người.

Mặt khác, tôi không nghĩ rằng những cư dân với tâm trạng không vui ấy chỉ vì mức sống thấp. Tôi cũng không nghĩ rằng họ mắc bệnh, hoặc họ là kẻ gây rối, hoặc có sự chênh lệch thu nhập đáng kể giữa cư dân nơi này. Phải có một lý do nào đó, cho cái tâm trạng khác thường này, nhưng tôi không biết là gì nếu chỉ nhìn.

"Liệu nó có liên quan đến chim lam... ?"

Ông lão hồi nãy nói rằng nhờ ơn chim lam, mọi người đều sống hạnh phúc.

Cho là vậy, thì có lẽ nhưng người kia không được chim lam ban phước ? Nếu vậy thì, tại sao ?

Tôi càng nghĩ, càng rối. Tôi cần thêm thông tin.

Tôi nhìn mặt trời coi còn bao nhiêu thời gian nữa. Vẫn còn nhiều. Đây chỉ là ngày thứ hai của tôi, nên chẳng ngạc nhiên khi tôi cần phải lục lọi lung tung về mọi thứ.

Và tôi làu bầu, rồi tiếp tục bước đi.

Tôi tự hỏi, chim lam to cỡ nào ? Tìm con chim bé nhỏ trong thị trấn rộng lớn thế này quả thật là một việc kinh khủng. Khi tìm ra, chắc tôi phải tăng giá cho bà buôn tin đội mũ. Với động lực ấy, tôi tiếp tục bước đi. \\


 \begin{center}
 	******
 \end{center}

% "Have you lived here a long time, miss?"
"Chị sống ở đây lâu chưa ?"

Chị ca sĩ suy nghĩ một hồi. Rồi nghiên đầu, đáp như đang nói "Chắc vậy ?"

"Chị ở đây một thời gian khá dài. Chị thấy hạnh phúc khi ở đây, nên, chị không định chuyển nhà đi đâu hết"

Chị ấy mỉn cười dịu dàng với mình. Nụ cười của chị ca sĩ thật đáng yêu. Chị ấy hát thật dễ thương, nhưng con người cũng thật sự dễ thương luôn.

"Em đến từ đâu, Efi ?"

Chị ấy hỏi bất ngờ quá.

"Mình từ đâu ra thế ?"

Mình nhớ khi mình vừa nở, mình thấy bố ... à, thấy Al, và Al trố to mắt nhìn mình. Sau đó, bọn mình đi rất nhiều chỗ, ví dụ như nơi này.

"uhh, nhiều nơi, nhiều nơi lắm !"

Chị ấy tỏ vẻ tò mò vì câu trả lời của tôi, nhưng lại mỉn cười ngay sau đó. 

Al bảo mình không được trả lời nếu ai đó hỏi về quá khứ, nên mình tự hỏi trả lời vậy có ổn không ? Efi luôn giữ lời hứa !

"Đúng thật, em là nhà lữ hành. Chị rất vui nếu em hài lòng với thị trấn này."

Chị ấy vừa nói xong thì cánh của kêu cạch cạch.

"Al"

Al xuất hiện sau cánh cửa. Vừa thấy tôi thì Al run lên vì lý do nào đó.

"Em có ngoan không ?"

"Dạ có!"

Chắc vậy. Tất cả những gì mình làm hôm nay là đi xung quanh căn biệt thự này, và được ngài quản gia nói "Anh thật bất ngờ khi em bỗng dưng biến mất". Rồi mình nói chuyện với chị ca sĩ, và khối người mà họ đã về nhà rồi. Vậy là vừa đủ "ngoan" !

"Em ngoan mà"

"Không cần phải nhắc lại thế đâu", Al ngãi đầu.

Chân mày của Al nhíu lại. Anh rất thường nhíu mày. Mình tự hỏi nó có nghĩa gì ?

Mình tự hỏi ... Có lẽ anh vẫn còn ngạc nhiên vì mình nghe lời anh ?

"Ehh !"

Mình nghĩ mình nên tự hào. Mình thở phào và thấy lông mày của Al còn nhíu lại hơn nữa.

"Ahhh". Mình ngáp một cái thật to. Mình đã đi bộ khá nhiều trong cái biệt thự này, lại còn nói chuyện với nhiều người nữa. Nên mình chắc là đã khá mệt.

"Ồ, hình như chúng tôi nói chuyện khá lâu, nên cũng trễ quá rồi. Efi đã rất ngoan, nên đừng lo. Tôi ghen tị với anh vì có em gái dễ thương đến vậy !!"

"Ồ, nó đã đẩy tôi vào khá nhiều rắc rối. Hahaha"

Al trả lời chị ấy với khuôn mặt và giọng nói khác trước. Al thật kì lạ. Mình nghĩ Al đang nói dối. Nhưng mình không thể nhìn xuyên thấu những lời nói dối của Al.

"Vậy, tôi xin phép. Chúc ngủ ngon."

"Chúc ngủ ngon !"

Mình vãy tay tạm biệt chị ca sĩ, và chị vẫy tay lại.

Khi chị ấy vừa đi khuất tầm mắt, Al trở lại con người thật thường ngày.

"... Về phòng nào."

Anh kéo tay mình với vẻ mệt mỏi, nên tôi vội chạy theo, cẩn thận kẻo té.\\

% Back in the room, first Al took off his coat, then jumped back onto the bad.

Về phòng, Al cởi áo khoát, và nhảy lên giường.

Mình rất thích nhảy lên giường, vì nệm rất êm và thoải mái. Mình rất vui khi thấy Al giống mình.

"À, tao quên đưa mày cái này. Chỉ có một cái vì tới giờ đi ngủ. Mày nên nghỉ ngơi một tí"

"Wow, cám ơn anh nhiều !"

Al cho tôi cái kẹo tròn được gói trong cái túi nhỏ xinh xinh. Tôi mở ra và bỏ vào mồm.

Lăn viên kẹo trên lưỡi, mình cảm nhận được vị ngọt. Vị ngọt không làm mình no, nhưng nó khiến mình thực sự vui. Và khi mình vui, Al mỉn cười một cách hài lòng. Mình thích như vậy.

"... ừm, chúng ta nên tóm tắt lại ngày hôm nay"

"Vâng ạ"

Mỗi khi tụi mình đi tới thị trấn điều tra gì đó và hành động độc lập, chúng mình thường báo cáo cho nhau những gì xảy ra trước khi chúc ngủ ngon.

Mình đã nói chuyện với ai, thấy gì ... Anh bảo mình kể mọi thứ, nên mình đã đi quanh biệt thự như thế nào, và mình đã nói chuyện với nữ ca sĩ.

"Mày thấy gì đáng chú ý khi khám phá biệt thự, hay nói chuyện với nữ ca sĩ không ?"

"Để trả lời tốt, mình nhắm mắt, và cố nhớ"

Biệt thự này rất ... rộng ? Nó chẳng lạ gì. Và chẳng có cái tượng của ông già lập dị như thị trấn trước.

Khi mình nói chuyện với chị ấy, ... Chị nói chị thích thị trấn này. Cũng chẳng lạ nhỉ ?

"ờ ...."

Nhưng mà, cái mà mình nghĩ là "lạ" khác với cái mà Al nghĩ, nên mình quyết định kể mọi thứ, như mọi lần. \\

% Once I was finished, Al nodded a few times. “...I see. Guess we can pretty much say we didn’t get much today.”

Khi tôi kể xong, Al gật đầu vài lần "... Xem nào. Tao đoán là ta chẳng tìm được gì nhiều hôm nay"

Tại vì mình nói nhiều quá, mình ngáp một cái thật to. Mình thực sự mệt ...

"Rồi, mày ngủ đi. Mai gặp lại"...

Al xoa đầu mình và chui vào chăn.

"Chúc ngủ ngon, Al !"

"Chúc ngủ ngon."

Chúng tôi nói câu cuối cùng trong ngày. Anh ấy luôn nói một cách siêu vô cảm, nhưng nếu bạn nói "Chào buổi sáng", thì Al cũng "Chào buổi sáng" lại. Và nếu bạn nói "Chúc ngủ ngon", Al cũng sẽ nói "Chúc ngủ ngon".

Tôi đoán Al là người tốt. Tôi gật đầu, và chui vào chăn, nhắm mắt.\\

Mình hy vọng rằng ngày mai cũng sẽ thế này. \\



