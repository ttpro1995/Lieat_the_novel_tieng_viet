% Source http://vgperson.tumblr.com/post/130927788020/lieat-the-novel-prologues-part-i
%   \section{The Lie-Eating Dragon and the Forgotten-Color Songstress}
%	\subsection*{Ngày 4}

Efi lay tôi dậy. Tôi thấy dạo này mình khó dậy sớm.

"Ừm, mấy giờ rồi nhỉ ?"

Đó là tất cả những gì tôi muốn biết ... Tôi lấy đồng hồ trong ba lô ra. Đã hơn 12 giờ rồi sao. Còn hơn cả dậy trễ.

Tôi kéo Efi rời phòng. Vừa mở cửa, giọng hát trong trẻo ở tầng dưới vang lên.

"Đó là chị ca sĩ."

"Chị ấy đang hát à ?"

"Vâng, cư dân đến đây mỗi ngày, và chị đó hát, và nói chuyện với họ mỗi ngày."

Tôi nhớ ngài quản gia Phil bảo như vậy lần đầu tôi đến. Nhưng thường giờ này, tôi đang ở ngoài, nên tôi không biết cô ta hát mỗi ngày. Chen ngang khi cô ấy hát để nói chuyện khi mọi người đang ở xung quanh thật bất lịch sự. 

Mặc dù mọi người trông vui vẻ và vô tư, chỉ hỏi thôi cũng có rủi ro bị đâm sau lưng. Ngoài ra, giọng hát của Mischa đang giúp tôi bình phục.

"Đi xuống nghe chị ấy hát nha !"

"Vâng"

Tôi nhìn xuống phòng tiệc từ tần hai, và im lặng lắng nghe giọng hát của Mischa. Bài này khác bài mà tôi nghe hôm chúng tôi đến.

"Giọng của chị ca sĩ hay tuyệt, nhỉ ?"

"Người ta không tự nhiên gọi bả là ca sĩ."

Tôi trả lời một cách chẳng hứng thú gì. Bài hát vừa tới khúc cao trào.

"... Bả có bao giờ hát lại bài nào không nhỉ ?"

"Ừm...", Efi trả lời một cách bối rối. Chắc là chưa.

"Ừm, em không nghĩ vậy."

"ờ"

Một câu trả lời không chắc chắn, nhưng Efi không phải là tên đần. Tôi ghi nhớ là cô ca sĩ ấy hát bài khác nhau mỗi ngày, rồi nói chuyện. \\

% "Oh, Al and Efi! Good day,"
"Ồ, Al, Efi, chúc một ngày tốt lành." Mischa chào và rót trà. "Các bạn có khỏe không ? Các bạn có muốn ăn trưa không ?". Cái bình trà lắc nhẹ trong tay Mischa.

Tôi nhẹ nhàng từ chối. "Tôi chỉ muốn nói chuyện một lát".

"Ngày tôi đến đây, tôi có mua thức ăn cho Efi, và nó muốn tôi mua lại món đó, nhưng mà tôi quên mất rồi. Nhưng mà, Mischa, chị ở đó cùng chúng tôi, tôi nghĩ chắc chị còn nhớ"

Mischa nhăn mặt. "Các bạn đến đây từ ... 3 ngày trước ?" Cô ta đáp ngập ngừng.

"Vâng, đúng vậy."

Tôi đợi Mischa trả lời. Trông cô ta như đang đăm chiêu suy nghĩ, rồi lặng lẽ cúi đầu.

"Xin lỗi ... tôi cũng không nhớ."

"Vậy sao ? Không, tôi xin lỗi vì đã làm phiền tiệt trà của chị."

Tôi cúi chào, và vỗ vai Efi. Nó kêu "a !" và theo tôi, vãy tay chào tạm biệt Mischa.

"Ta sẽ đi đâu ?"

"Gặp ngài quản gia"

Efi kể ngài quản gia và cô ca sĩ đã từng là bạn tốt lâu năm. Vậy chắc chắn là anh ta sẽ biết gì đó về Mischa.

Tôi đang cân nhắc hai khả năng. Một là vì lý do gì đó, Mischa bị mất trí nhớ.

Efi đã ở cạnh Mischa suốt mấy ngày qua. Nó chẳng nói đã chứng kiến sự cố nào cả, nên cái "lý do gì đó" ngoài tầm kiểm soát của tôi. Nhưng có thể, chỉ là có khả năng là, nó có liên quan với lý do tại sao cư dân hành động kì lạ như thế.

Khả năng còn lại là Mischa lần đầu tôi gặp và Mischa bây giờ là hai người khác nhau.

Trường hợp này, thì tôi thấy nó liên quan đến chuyện ma mà cư dân buồn bã nói, như là "thấy ông nội đã chết."

Cá nhân tôi, không thể hình dung ra được hai Mischa khác nhau. Tôi không biết ai là thật, nhưng Mischa thật ít nhất không có lý do để nói dối, và nếu cô ta nói dối thì rồng ăn nói dối Efi sẽ biết.

"Ngài quản gia kìa"

Efi kéo tôi ra khỏi dòng suy nghĩ. Tôi nhìn về hướng nó chỉ. Phil đang dọn vườn.

Anh dừng lại và nhìn chúng tôi: "Ồ, chúc một ngày tốt lành. Có chuyện gì vậy ? Các bạn lại đi lạc nữa à ?"

"Không, hôm nay em không lạc đâu", Efi đáp.

Làm sao mà có thể lạc được khi đi từ căn biệt thự ra đây ? Mặc dù nó khá xa phòng tiệc... Nhưng giờ không phải lúc hỏi.

Tôi phá tan bầu không khí yên bình giữa hai người và hỏi Phil.

"Xin lỗi, chúng ta có thể nói chuyện một chốc không ?"

\begin{center}
	******
\end{center}

Tớ sẽ bảo vệ cậu, để cậu có thể cười và sống hạnh phúc.

Ngay từ khi chúng ta còn nhỏ, tớ dành cho cậu mọi thứ. Chẳng có lý do gì để tớ làm vậy. Chỉ vì, tớ muốn vậy.

Tôi bỗng ngẫm nghĩ khi tôi đang làm vườn.

Có ổn khi để mọi chuyện như thế này ?

Trước khi tôi có thể tự trả lời. Tôi nghe tiếng bước chân từ đằng sau. Tôi quay lại và thấy hai anh em lữ hành đến đây từ vài ngày trước, họ là Al và Efi. \\

"...Có chuyện gì thế ?"

Có vẻ như hai anh em này đang tìm gì đó kể từ khi họ đến thị trấn này. Mạch tôi bỗng đập nhanh.

"Về cô Mischa, ... xin tha lỗi cho tôi, nhưng gần đây, có phải cô ấy bị đãng trí ?"

Bỗng tôi được hỏi về cái mà tôi chẳng muốn, tôi cứng họng. Tôi nên trả lời thế nào đây ?

"Tôi hỏi về việc mà cô ấy chứng kiến ngày đầu chúng tôi đến biệt thự. Nhưng cô ta hoàn toàn không biết. Chỉ việc này không thì có vẻ như cô ấy bất cẩn... Nhưng ngoài ra, vài đoạn đối thoại khác lại làm tôi lo lắng"

"... Như là ?"

"Cụ thể ... Mischa làm như là vừa mới biết việc đó lần đầu mặc dù đáng lẽ cô đã biết. Nhưng cô ấy không có ý định chọc hay lừa chúng tôi. Nên tôi nghĩ có gì đó đã ảnh hướng trí nhớ của cô"

Tôi thấy khó chịu và mất kiên nhẫn "... Các bạn đã hỏi Mischa về việc đó chưa ?"

Việc này thật tệ. Không được nói Mischa là cậu ấy mất trí nhớ. Bởi vì Mischa sẽ rất buồn.

Và không chỉ duy nhất một mình cậu ấy, còn ...

"Không, tất nhiên là tôi không thể hỏi cô ấy "Này, chị còn nhớ gì không ?", tôi hỏi đủ để tự suy ra việc này"

Không hề để ý rằng tôi đang im lặng, Al tiếp tục nói. Tôi giữ bình tình. Để tôi không làm lộ bất kì thông tin gì.

"Để xem ... Cô ấy có bị đập vào đầu ? Hay là bị bệnh, hoặc di truyền dẫn đến việc đãng trí từ nhỏ. Hoặc có ai đó cố tình xóa trí nhớ cô ây ? Anh có biết gì không Phil ?"

"Có phải đây có liên quan đến việc các bạn đang điều tra về thị trấn này ?"

Al gật đầu "Vâng"

Tôi nghĩ sẽ rắc rối nếu tôi hỏi trực tiếp họ đang điều tra về gì. Nhưng tôi chắc chắn nó liên quan đến Mischa.

"Vậy thì ... vâng, có thể là các bạn đã gặp Charlotte."

"Charlotte ... ?" Al tròn xoe mắt bởi vì câu trả lời nằm ngoài dự kiến của tôi.

Vâng, đúng vậy, hoặc là, không. Tôi cũng không ngờ. Đó là lời nói dối ngẫu nhiên từ miệng tôi.

"Mischa có một người em song sinh thỉnh thoảng đến thăm chúng tôi"

Đã nói đến nước này rồi thì tôi không thể dừng. Tôi không thể không cảm thấy ấy nấy vì đã nói dối.

Nhưng tôi không có sự lựa chọn nào khác.

Người tôi càng thấy ấy nấy, nặng nề. Đợi đã, hình như ...
Nó nặng thật ?

"Oa !"

Cái vật màu đen bám vào chân tôi. Hình dáng như một con mèo dễ thương được bao phủ bởi cái gì đó trông như bùn đen.

Tôi cố gắn hất nó ra một cách tuyệt vọng, nhưng con quái vật chẳng nhút nhích.

"Vậy là, anh nói dối"

Tôi nhìn Al.

"Đừng lo, chỉ là lời nói dối nhỏ. Nó không cắn đâu, miễn là anh không nói dối nữa."

"Em ăn nó nha ?", Efi mắt long lanh hỏi vọng lên từ sau Al.

"Hả ? Mày đói rồi à ?"

"Em muốn ăn nó !"

"Được rồi, được rồi, ăn đi"

Nói xong, Efi nhảy vồ vào chân tôi, mạnh đến nỗi tôi té nhào.

"À, xin lỗi. Hãy nằm như vậy và để tôi nói"

Tôi không thể từ chối. Con quái vậy bám chặt vào chân tôi, tôi chẳng thể chạy. Tôi gật đầu một cách yếu ớt.

"Được rồi ... tôi sẽ giải thích con quái vật trên chân anh, nhưng trước tiên. tôi sẽ giải thích về Efi. Efi là rồng. Anh có biết gì về rồng ?"

"Thường thì, tôi biết. Một sinh vật có năng lực kì lạ từ khi được sinh ra, đúng không ?"

"Đúng. Trường hợp của Efi là ăn nói dối. Khi ai đó nói dối, nó biến lời nói dối thành vật chất rồi ăn. Đó là con quái vật trên chân anh."

Tôi nhìn lại chân mình, con quái vật mất rồi. Thay vào đó, nó đang trong tay của Efi. Nó nhồi hai gò má vào con quái vật, với vẻ biểu cảm như đang ăn một bữa ngon lành. Chắc chắn nó đang ăn con quái vật.

"Vâng có vẻ như anh hiểu. Vậy tiếp theo, tại sao anh nói dối ?"

Tôi cảm thấy một cái nhìn sắc lẹm. Vậy là anh ta đã nhìn xuyên thấu mọi thứ.

"Tôi xin lỗi..."

Tôi không thể nói dối trước mặt những người này. Tôi biết rõ điều này.

"Mischa... bị bệnh từ nhỏ. Mọi kí ức của cậu ấy sẽ biến mất sau ba ngày."

Tôi chậm rãi kể cho họ về Mischa. Al lắng nghe một cách nghiêm túc.

"Cậu ấy có vẻ nhớ những gì quan trọng, như tên cậu ấy, gia đình, cậu ấy thích hát. Những kí ức khác, thậm chí về tôi, cậu ấy quên trong ba ngày."

Giọng tôi rưng rưng. Tôi đã hiểu điều này từ lâu, nhưng mà ngực tôi bỗng thắt lại.

"Nhưng ... nếu cậu ấy biết. Nếu ai đó nói là, trong ba ngày, cậu ta sẽ quên mọi thứ, kể cả căn bệnh ấy ... cậu ta sẽ rất buồn. Tôi đã từng nói một lần, đã lâu ... và tôi không bao giờ muốn Mischa buồn như vậy nữa. Nên ... anh giữ kín chuyện này nhé ?" Tôi cúi đầu, van nài.

Tôi đang làm gì thế này ? Nói dối họ, xong rồi cầu xin ? Nhưng tôi bị dồn vào chân tường rồi.

Tôi nghe giọng tử tế hơn tôi mong đợi.

"Thật ra, sự thật Efi là rồng mọi người cũng không nên biết, vì ... thứ hiếm như vậy sẽ bị săn lùn, anh biết ấy. Và tôi thì, cũng bị dồn vào tình huống không thể giấu. Nên, hãy cùng giữ bí mật chuyện này, nhé."

Al cười méo mặt, và cũng nhờ vả tôi.

Tâm trạng anh ta thay đổi chóng mặt so với cách anh nghiêm khắc tra hỏi tôi hồi nãy. Tôi tự hỏi anh ta đang giấu một cái gì đó còn lớn hơn.

"... Tôi hiểu"

Tôi đang ngồi một cách nhục nhã dưới đất. Al kéo tôi dậy.

"Anh quản gia, anh sẽ không nói dối nữa chứ ?"

Đột nhiên, Efi như cắn vào chỗ đau, và tôi cảm thấy vừa hối hận, vừa mệt mỏi. Có vẻ Efi vừa ăn xong, vô vỗ cái bụng no tròn.

Nó ăn nói dối. Rồng quả thật là sinh vật kì lạ.

"Anh sẽ không nói dối nữa đâu. Anh vừa biết là không thể nói dối trước mặt em."

Efi làm vẻ mặt có vẻ thấy vọng. Nó đang mong chờ điều gì à ?.

"... Chào tạm biệt, chúng tôi còn có việc phải đi điều tra."

Al cúi chào và mỉn cười. Tôi xin lỗi lần nữa và cúi chào tạm biệt.

% .Are you okay? That looked like it was tough

"... Anh có sao không ? Mọi thứ trông vất vả nhỉ."

Một khi Al và Efi đi khuất tầm mắt, một giọng nói từ sau lưng tôi.

Khi tôi quay lạ, đó là chim lam.

"Chim lam ..."

Mặc dù tôi gọi nó là chim lam, nó là một con người với cái cánh chim thật to sau lưng. Tôi chẳng quen với cái tên đó.

"Sao anh không nói gì hết vậy ?"

"Chẳng phải em đã từng nói sự hiện diện của em không nên để con người biết sao?"

"Em nói vậy à ?"

Nó giả ngây, và mỉn cười một cách điềm tĩnh. Đôi cánh xanh thẫm trên lưng vỗ nhẹ. \\

% My meeting with the bluebird occurred 

Tôi gặp chim lam lần đầu khi tôi đang dọn nhà kho biệt thự. 

Thực ra, nó không trông giống thế này. Khi tôi tìm thấy nó, nó chỉ là quả trứng khổng lồ.

Trứng gì thế này, tôi tự hỏi ? Nếu tôi để nó ở đây, chắc nó vỡ mất. Nên tôi cầm nó lên một cách cẩn trọng - và lập tức, một đường nứt chạy dài trên vỏ trứng.

Ồ, không. Tôi cầm mạnh quá ? Cái vết nứt lan rộng khắp quả trứng. Và rồi, tôi thấy khuôn mặt của một sinh vật giống như một bé gái bên trong trứng. 

Tôi tả nó là "giống như" một bé gái, vì nó có thêm đôi cánh to sau lưng. Đôi cách tuyệt đẹp ấy như chứa đựng cả một bầu trời. Trong vài giây, tôi như bị hóp hồn bởi vẻ đẹp của đôi cánh ấy.

"Em là ai ?". Tôi hỏi, khi đang cố gắn tìm hiểu xem đứa trẻ từ đâu ra.

"Hở, em ? hừm, em là ai ?"

Nó trả lời bằng một câu hỏi khác. Chẳng biết làm sao, tôi méo mặt.

"Ồ, em chỉ biết nhiêu đây. Em đến đây để ban cho anh điều ước."

"Hở?"

Ban cho tôi ... điều ước ?

"Vậy... chắc em là "Chim Lam" ?"

Có một truyền thuyết trong thị trấn này. Nó kể về một sinh vật huyền bí có thể biến bất kì điều ước gì thành hiện thực.

Thấy đôi cánh màu xanh, tôi nghĩ, chính là nó.

"Nếu anh gọi em như vậy."

Một câu trả lời khó hiểu. Nhưng đối với tôi, chẳng quan trọng. Thứ quan trọng là nó ban cho tôi điều ước.

"Vâng, việc này ... hơi đột ngột nhưng mà,... anh có thể ước luôn được không ?"

"Vâng, anh muốn ước gì ?"

Tôi chỉ có một điều ước.

"... Anh có một người bạn tên Mischa. Anh muốn chữa bệnh cho bạn ấy."

"Hừm..."

Có thật là nó sẽ ban cho tôi điều ước ? Nó có đòi trả tiền không ? Mặc dù lúc đầu tôi nói điều ước này một cách kiên định, giờ tôi bắt đầu thấy lo.

Nhưng mà, chim lam lắc đầu "Em không nghĩ năng lực của em đủ thực hiện điều ước này."

Hả ... ? Vậy nó không thực hiện được điều ước nào hết ? Tôi thấy thất vọng. Nó không để ý việc này và tiếp tục hỏi.

"Anh cần gì nữa không ?"

Tôi vừa xin điều ước mà tôi canh cánh bấy lâu nay. Ngoài ra tôi chẳng nghĩ ra gì cả. Tôi chỉ ước cho Mischa... Lúc này, tôi bỗng nảy ra một ý.

"À, ...vậy thì, Mischa sẽ hạnh phúc. Anh muốn Mischa sẽ không bao giờ buồn, và luôn mỉn cười hạnh phúc."

Tôi vừa nói xong thì, tôi chợt nhận ra điều ước này khá mơ hồ. Chắc không được đâu.

"Àaaaaa, em có thể thực hiện điều ước này. Chắc chắn."

"Hở ?"

Thiệt không ? Tôi nghe thật bất ngờ. Tôi bắt đầu thấy lo, liệu mọi thứ có ổn không.

Rồi, chim lam xòe cánh nó ra, sải cánh của nó vừa đủ căn phòng. Rồi một ánh sáng chói lòa lan khắp phòng.

"Điều ước của anh đã thành hiện thực"

Tôi rụt rè mở mắt. Chim lam vân đứng đó, với biểu cảm như cũ.

"...Xong chưa ?"

"Vâng, xong cả rồi."

Thật thất vọng. Tôi chưa ra ngoài, tôi chưa biết mọi thứ đã thay đổi như thế nào.

"Điều ước của anh "có lẽ" đã thành hiện thực. ", chim lam nói, kèm theo từ "có lẽ" làm tôi thấy bất an.

"Em là chim lam. Em nên gọi anh là gì ?"

Nghĩ lại thì, tôi chưa nói tên mình cho nó.

"Ờ..."

"Phil"

Tôi ngẩn đầu và nhìn vào chim lam.

"...Em đã thực hiện điều ước của anh rồi chứ, phải không ?"

"Nếu anh tin vậy."

Chim lam lại đáp một cách không rõ ràng.

"Anh đang nghĩ gì vậy ?"

"Chỉ là anh hồi tưởng lại sự việc mà chúng ta gặp nhau."

"Ờ phải ... Em đang trần truồng, anh quả thật là tệ hại."

... Tôi chẳng muốn nói chuyện này chút nào.

"Mọi thứ liệu có ổn không, ngay cả khi anh không thể hạnh phúc, Phil ?"

"Anh ổn cả. Miễn là Mischa vui, thì đó cũng là niềm hạnh phúc của anh luôn... À, vâng, anh đang lau dọn. Anh sẽ tiếp tục công việc"

Tôi nhặt cái chổi lên. Đúng. Ngay cả khi mình không bao giờ hạnh phúc, mọi thứ đều ổn.

\begin{center}
	******
\end{center}

% What're we gonna do now